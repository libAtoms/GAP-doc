%% Generated by Sphinx.
\def\sphinxdocclass{report}
\documentclass[letterpaper,10pt,english]{sphinxmanual}
\ifdefined\pdfpxdimen
   \let\sphinxpxdimen\pdfpxdimen\else\newdimen\sphinxpxdimen
\fi \sphinxpxdimen=.75bp\relax

\PassOptionsToPackage{warn}{textcomp}
\usepackage[utf8]{inputenc}
\ifdefined\DeclareUnicodeCharacter
% support both utf8 and utf8x syntaxes
\edef\sphinxdqmaybe{\ifdefined\DeclareUnicodeCharacterAsOptional\string"\fi}
  \DeclareUnicodeCharacter{\sphinxdqmaybe00A0}{\nobreakspace}
  \DeclareUnicodeCharacter{\sphinxdqmaybe2500}{\sphinxunichar{2500}}
  \DeclareUnicodeCharacter{\sphinxdqmaybe2502}{\sphinxunichar{2502}}
  \DeclareUnicodeCharacter{\sphinxdqmaybe2514}{\sphinxunichar{2514}}
  \DeclareUnicodeCharacter{\sphinxdqmaybe251C}{\sphinxunichar{251C}}
  \DeclareUnicodeCharacter{\sphinxdqmaybe2572}{\textbackslash}
\fi
\usepackage{cmap}
\usepackage[T1]{fontenc}
\usepackage{amsmath,amssymb,amstext}
\usepackage{babel}
\usepackage{times}
\usepackage[Bjarne]{fncychap}
\usepackage{sphinx}

\fvset{fontsize=\small}
\usepackage{geometry}

% Include hyperref last.
\usepackage{hyperref}
% Fix anchor placement for figures with captions.
\usepackage{hypcap}% it must be loaded after hyperref.
% Set up styles of URL: it should be placed after hyperref.
\urlstyle{same}
\addto\captionsenglish{\renewcommand{\contentsname}{Contents:}}

\addto\captionsenglish{\renewcommand{\figurename}{Fig.}}
\addto\captionsenglish{\renewcommand{\tablename}{Table}}
\addto\captionsenglish{\renewcommand{\literalblockname}{Listing}}

\addto\captionsenglish{\renewcommand{\literalblockcontinuedname}{continued from previous page}}
\addto\captionsenglish{\renewcommand{\literalblockcontinuesname}{continues on next page}}
\addto\captionsenglish{\renewcommand{\sphinxnonalphabeticalgroupname}{Non-alphabetical}}
\addto\captionsenglish{\renewcommand{\sphinxsymbolsname}{Symbols}}
\addto\captionsenglish{\renewcommand{\sphinxnumbersname}{Numbers}}

\addto\extrasenglish{\def\pageautorefname{page}}

\setcounter{tocdepth}{3}
\setcounter{secnumdepth}{3}


\title{GAP Wrapper Documentation}
\date{Jan 18, 2019}
\release{0.1}
\author{Adam Fekete}
\newcommand{\sphinxlogo}{\vbox{}}
\renewcommand{\releasename}{Release}
\makeindex
\begin{document}

\pagestyle{empty}
\sphinxmaketitle
\pagestyle{plain}
\sphinxtableofcontents
\pagestyle{normal}
\phantomsection\label{\detokenize{index::doc}}\phantomsection\label{\detokenize{index:module-quippy}}\index{quippy (module)@\spxentry{quippy}\spxextra{module}}


The \sphinxcode{\sphinxupquote{QUIP}} package (\sphinxhref{https://github.com/libAtoms/QUIP}{GitHub}) is a
collection of software tools to carry out molecular dynamics
simulations. It implements a variety of interatomic potentials and
tight binding quantum mechanics, and is also able to call external
packages, and serve as plugins to other software such as \sphinxhref{http://lammps.sandia.gov}{LAMMPS}, \sphinxhref{http://www.cp2k.org}{CP2K} and also
the python framework \sphinxhref{https://wiki.fysik.dtu.dk/ase}{ASE}.  Various
hybrid combinations are also supported in the style of QM/MM, with a
particular focus on materials systems such as metals and
semiconductors.

\sphinxcode{\sphinxupquote{quippy}} is a Python interface to \sphinxcode{\sphinxupquote{QUIP}} that provides deep access to
most of the Fortran types and routines. The quippy interface is principally
maintained by \sphinxhref{http://www.warwick.ac.uk/jrkermode}{James Kermode}.
\begin{description}
\item[{Long term support of the package is ensured by:}] \leavevmode\begin{itemize}
\item {} 
Noam Bernstein (Naval Research Laboratory)

\item {} 
Gabor Csanyi (University of Cambridge)

\item {} 
James Kermode (University of Warwick)

\end{itemize}

\end{description}

Portions of this code were written by: Albert Bartok-Partay, Livia
Bartok-Partay, Federico Bianchini, Anke Butenuth, Marco Caccin,
Silvia Cereda, Gabor Csanyi, Alessio Comisso, Tom Daff, ST John,
Chiara Gattinoni, Gianpietro Moras, James Kermode, Letif Mones,
Alan Nichol, David Packwood, Lars Pastewka, Giovanni Peralta, Ivan
Solt, Oliver Strickson, Wojciech Szlachta, Csilla Varnai, Steven
Winfield.

Copyright 2006-2016.

Most of the publicly available version is released under the GNU
General Public license, version 2, with some portions in the public
domain.


\chapter{Overview of \sphinxstyleliteralintitle{\sphinxupquote{libAtoms}} and \sphinxstyleliteralintitle{\sphinxupquote{QUIP}}}
\label{\detokenize{index:overview-of-libatoms-and-quip}}\begin{itemize}
\item {} 
The \sphinxhref{http://www.libatoms.org}{libAtoms} package is a software
library written in Fortran 95 for the purposes of carrying out
molecular dynamics simulations.

\item {} 
The \sphinxcode{\sphinxupquote{QUIP}} (\sphinxstylestrong{QU}antum mechanics and \sphinxstylestrong{I}nteratomic
\sphinxstylestrong{P}otentials) package, built on top of \sphinxcode{\sphinxupquote{libAtoms}}, implements a
wide variety of interatomic potentials and tight binding quantum
mechanics, and is also able to call external packages.

\item {} 
Various hybrid combinations are also supported in the style of
QM/MM, including \sphinxtitleref{Learn on the Fly} scheme \sphinxcite{index:lotf}

\item {} 
\sphinxhref{http://www.jrkermode.co.uk/quippy}{quippy} is a Python interface
to libAtoms and QUIP.

\end{itemize}


\chapter{Features}
\label{\detokenize{index:features}}
The following interatomic potentials are presently coded or linked in QUIP:
\begin{itemize}
\item {} 
EAM (fcc metals)

\item {} 
Fanourgakis-Xantheas (water)

\item {} 
Finnis-Sinclair (bcc metals)

\item {} 
Flikkema-Bromley

\item {} 
GAP (Gaussian Approximation Potentials: general many-body)

\item {} 
Guggenheim-!McGlashan

\item {} 
Brenner (carbon)

\item {} 
OpenKIM (general interface)

\item {} 
Lennard-Jones

\item {} 
Morse

\item {} 
Partridge-Schwenke (water monomer)

\item {} 
Stillinger-Weber (carbon, silicon, germanium)

\item {} 
SiMEAM (silicon)

\item {} 
Sutton-Chen

\item {} 
Tangney-Scandolo (silica, titania etc)

\item {} 
Tersoff (silicon, carbon)

\end{itemize}

The following tight-binding functional forms and parametrisations are implemented:
\begin{itemize}
\item {} 
Bowler

\item {} 
DFTB

\item {} 
GSP

\item {} 
NRL-TB

\end{itemize}

The following external packages can be called:
\begin{itemize}
\item {} 
CASTEP

\item {} 
VASP

\item {} 
CP2K

\item {} 
ASAP

\item {} 
ASE (recent version, 3.11+, recommended)

\item {} 
Molpro

\end{itemize}


\section{Introduction}
\label{\detokenize{introduction:introduction}}\label{\detokenize{introduction::doc}}

\subsection{Overview}
\label{\detokenize{introduction:overview}}

\subsection{Installation}
\label{\detokenize{introduction:installation}}

\subsection{First exmaple}
\label{\detokenize{introduction:first-exmaple}}\phantomsection\label{\detokenize{gap:module-gap}}\index{gap (module)@\spxentry{gap}\spxextra{module}}

\section{Learning GAP}
\label{\detokenize{gap:learning-gap}}\label{\detokenize{gap::doc}}
Python wrapper for the teach\_sparse program


\subsection{Gaussian Approximation Potentials (GAP)}
\label{\detokenize{gap:gaussian-approximation-potentials-gap}}\index{GAP (class in gap)@\spxentry{GAP}\spxextra{class in gap}}

\begin{fulllineitems}
\phantomsection\label{\detokenize{gap:gap.GAP}}\pysiglinewithargsret{\sphinxbfcode{\sphinxupquote{class }}\sphinxcode{\sphinxupquote{gap.}}\sphinxbfcode{\sphinxupquote{GAP}}}{\emph{gap}, \emph{default\_sigma}, \emph{config\_type\_sigma=None}, \emph{core\_ip\_args=None}, \emph{core\_param\_file='quip\_params.xml'}, \emph{do\_e0\_avg=True}, \emph{do\_ip\_timing=False}, \emph{e0='0'}, \emph{e0\_offset=0.0}, \emph{hessian\_delta=1.0}, \emph{sigma\_parameter\_name='sigma'}, \emph{sigma\_per\_atom=True}, \emph{sparse\_jitter=1.0}, \emph{sparse\_use\_actual\_gpcov=False}, \emph{template\_file='template.xml'}, \emph{verbosity=\textless{}Verbose.Normal: 'NORMAL'\textgreater{}}}{}
GAP Wrapper
This subroutine parses the main command line options.
\begin{quote}\begin{description}
\item[{Parameters}] \leavevmode\begin{itemize}
\item {} 
\sphinxstyleliteralstrong{\sphinxupquote{gap}} (\sphinxstyleliteralemphasis{\sphinxupquote{list of Descriptors}}) \textendash{} Initialisation string for GAPs

\item {} 
\sphinxstyleliteralstrong{\sphinxupquote{default\_sigma}} (\sphinxstyleliteralemphasis{\sphinxupquote{float}}) \textendash{} Error in {[}energies forces virials hessians{]}

\item {} 
\sphinxstyleliteralstrong{\sphinxupquote{config\_type\_sigma}} (\sphinxstyleliteralemphasis{\sphinxupquote{str}}) \textendash{} What sigma values to choose for each type of data. Format:
\{type:energy:force:virial:hessian\}

\item {} 
\sphinxstyleliteralstrong{\sphinxupquote{core\_ip\_args}} (\sphinxstyleliteralemphasis{\sphinxupquote{str}}) \textendash{} QUIP init string for a potential to subtract from data (and added back after prediction)

\item {} 
\sphinxstyleliteralstrong{\sphinxupquote{core\_param\_file}} (\sphinxstyleliteralemphasis{\sphinxupquote{str}}) \textendash{} QUIP XML file for a potential to subtract from data (and added back after prediction)

\item {} 
\sphinxstyleliteralstrong{\sphinxupquote{do\_e0\_avg}} (\sphinxstyleliteralemphasis{\sphinxupquote{bool}}) \textendash{} Method of calculating e0 if not explicitly specified. If true, computes the average atomic
energy in input data. If false, sets e0 to the lowest atomic energy in the input data.

\item {} 
\sphinxstyleliteralstrong{\sphinxupquote{do\_ip\_timing}} (\sphinxstyleliteralemphasis{\sphinxupquote{bool}}) \textendash{} To enable or not timing of the interatomic potential.

\item {} 
\sphinxstyleliteralstrong{\sphinxupquote{e0}} (\sphinxstyleliteralemphasis{\sphinxupquote{str}}) \textendash{} Atomic energy value to be subtracted from energies before fitting (and added back on after
prediction). Specifiy a single number (used for all species) or by species: \{Ti:-150.0:O:-320\}.
energy = core + GAP + e0

\item {} 
\sphinxstyleliteralstrong{\sphinxupquote{e0\_offset}} (\sphinxstyleliteralemphasis{\sphinxupquote{float}}) \textendash{} Offset of baseline. If zero, the offset is the average atomic energy of the input data or
the e0 specified manually.

\item {} 
\sphinxstyleliteralstrong{\sphinxupquote{hessian\_delta}} (\sphinxstyleliteralemphasis{\sphinxupquote{float}}) \textendash{} Delta to use in numerical differentiation when obtaining second derivative for the
Hessian covariance

\item {} 
\sphinxstyleliteralstrong{\sphinxupquote{sigma\_parameter\_name}} (\sphinxstyleliteralemphasis{\sphinxupquote{str}}) \textendash{} Sigma parameters (error hyper) for a given configuration in the database. Overrides
the command line sigmas. In the XYZ, it must be prepended by energy\_, force\_, virial\_ or hessian\_

\item {} 
\sphinxstyleliteralstrong{\sphinxupquote{sigma\_per\_atom}} (\sphinxstyleliteralemphasis{\sphinxupquote{bool}}) \textendash{} Interpretation of the energy and virial sigmas specified in \textgreater{}\textgreater{}default\_sigma\textless{}\textless{} and
\textgreater{}\textgreater{}config\_type\_sigma\textless{}\textless{}. If \textgreater{}\textgreater{}T\textless{}\textless{}, they are interpreted as per-atom errors, and the variance will be scaled
according to the number of atoms in the configuration. If \textgreater{}\textgreater{}F\textless{}\textless{} they are treated as absolute errors and no
scaling is performed. NOTE: sigmas specified on a per-configuration basis (see \textgreater{}\textgreater{}sigma\_parameter\_name\textless{}\textless{})
are always absolute.

\item {} 
\sphinxstyleliteralstrong{\sphinxupquote{sparse\_jitter}} (\sphinxstyleliteralemphasis{\sphinxupquote{float}}) \textendash{} Intrisic error of atomic/bond energy, used to regularise the sparse covariance matrix

\item {} 
\sphinxstyleliteralstrong{\sphinxupquote{sparse\_use\_actual\_gpcov}} (\sphinxstyleliteralemphasis{\sphinxupquote{bool}}) \textendash{} Use actual GP covariance for sparsification methods

\item {} 
\sphinxstyleliteralstrong{\sphinxupquote{template\_file}} (\sphinxstyleliteralemphasis{\sphinxupquote{str}}) \textendash{} Template XYZ file for initialising object

\item {} 
\sphinxstyleliteralstrong{\sphinxupquote{verbosity}} ({\hyperref[\detokenize{gap:gap.Verbose}]{\sphinxcrossref{\sphinxstyleliteralemphasis{\sphinxupquote{Verbose}}}}}) \textendash{} Verbosity control.

\end{itemize}

\end{description}\end{quote}
\index{teach() (gap.GAP method)@\spxentry{teach()}\spxextra{gap.GAP method}}

\begin{fulllineitems}
\phantomsection\label{\detokenize{gap:gap.GAP.teach}}\pysiglinewithargsret{\sphinxbfcode{\sphinxupquote{teach}}}{\emph{at\_file}, \emph{gp\_file='gp\_new.xml'}, \emph{config\_type\_parameter\_name='config\_type'}, \emph{energy\_parameter\_name='energy'}, \emph{force\_parameter\_name='force'}, \emph{hessian\_parameter\_name='hessian'}, \emph{virial\_parameter\_name='virial'}, \emph{do\_copy\_at\_file=True}, \emph{sparse\_separate\_file=True}, \emph{rnd\_seed=-1}}{}~\begin{quote}\begin{description}
\item[{Parameters}] \leavevmode\begin{itemize}
\item {} 
\sphinxstyleliteralstrong{\sphinxupquote{at\_file}} (\sphinxstyleliteralemphasis{\sphinxupquote{str}}) \textendash{} XYZ file with teaching configurations

\item {} 
\sphinxstyleliteralstrong{\sphinxupquote{gp\_file}} (\sphinxstyleliteralemphasis{\sphinxupquote{str}}) \textendash{} Output XML file

\item {} 
\sphinxstyleliteralstrong{\sphinxupquote{config\_type\_parameter\_name}} (\sphinxstyleliteralemphasis{\sphinxupquote{str}}) \textendash{} Identifier of property determining the type of input data in the at\_file

\item {} 
\sphinxstyleliteralstrong{\sphinxupquote{do\_copy\_at\_file}} (\sphinxstyleliteralemphasis{\sphinxupquote{bool}}) \textendash{} Do copy the at\_file into the GAP XML file (should be set to False for NetCDF input).

\item {} 
\sphinxstyleliteralstrong{\sphinxupquote{energy\_parameter\_name}} (\sphinxstyleliteralemphasis{\sphinxupquote{str}}) \textendash{} Name of energy property in the at\_file that describes the data

\item {} 
\sphinxstyleliteralstrong{\sphinxupquote{force\_parameter\_name}} (\sphinxstyleliteralemphasis{\sphinxupquote{str}}) \textendash{} Name of force property in the at\_file that describes the data

\item {} 
\sphinxstyleliteralstrong{\sphinxupquote{hessian\_parameter\_name}} (\sphinxstyleliteralemphasis{\sphinxupquote{str}}) \textendash{} Name of hessian property in the at\_file that describes the data

\item {} 
\sphinxstyleliteralstrong{\sphinxupquote{rnd\_seed}} (\sphinxstyleliteralemphasis{\sphinxupquote{int}}) \textendash{} Random seed.

\item {} 
\sphinxstyleliteralstrong{\sphinxupquote{sparse\_separate\_file}} (\sphinxstyleliteralemphasis{\sphinxupquote{bool}}) \textendash{} Save sparse coordinates data in separate file

\item {} 
\sphinxstyleliteralstrong{\sphinxupquote{virial\_parameter\_name}} (\sphinxstyleliteralemphasis{\sphinxupquote{str}}) \textendash{} Name of virial property in the at\_file that describes the data

\end{itemize}

\end{description}\end{quote}

\end{fulllineitems}


\end{fulllineitems}



\subsection{Additional helper}
\label{\detokenize{gap:additional-helper}}\index{Verbose (class in gap)@\spxentry{Verbose}\spxextra{class in gap}}

\begin{fulllineitems}
\phantomsection\label{\detokenize{gap:gap.Verbose}}\pysigline{\sphinxbfcode{\sphinxupquote{class }}\sphinxcode{\sphinxupquote{gap.}}\sphinxbfcode{\sphinxupquote{Verbose}}}
Verbosity control. Options:
\begin{itemize}
\item {} 
\sphinxstylestrong{NORMAL},

\item {} 
\sphinxstylestrong{VERBOSE},

\item {} 
\sphinxstylestrong{NERD},

\item {} 
\sphinxstylestrong{ANAL}.

\end{itemize}

\end{fulllineitems}



\section{Descriptors}
\label{\detokenize{descriptors:descriptors}}\label{\detokenize{descriptors::doc}}
Descriptors for atomic environments


\subsection{List of descriptors}
\label{\detokenize{descriptors:module-gap.descriptors}}\label{\detokenize{descriptors:list-of-descriptors}}\index{gap.descriptors (module)@\spxentry{gap.descriptors}\spxextra{module}}\index{A2\_dimer (class in gap.descriptors)@\spxentry{A2\_dimer}\spxextra{class in gap.descriptors}}

\begin{fulllineitems}
\phantomsection\label{\detokenize{descriptors:gap.descriptors.A2_dimer}}\pysiglinewithargsret{\sphinxbfcode{\sphinxupquote{class }}\sphinxcode{\sphinxupquote{gap.descriptors.}}\sphinxbfcode{\sphinxupquote{A2\_dimer}}}{\emph{cutoff=0.0}, \emph{monomer\_cutoff=1.5}, \emph{atomic\_number=1}}{}~\begin{quote}\begin{description}
\item[{Parameters}] \leavevmode\begin{itemize}
\item {} 
\sphinxstyleliteralstrong{\sphinxupquote{cutoff}} (\sphinxstyleliteralemphasis{\sphinxupquote{float}}) \textendash{} Cutoff for A2\_dimer-type descriptors

\item {} 
\sphinxstyleliteralstrong{\sphinxupquote{monomer\_cutoff}} (\sphinxstyleliteralemphasis{\sphinxupquote{float}}) \textendash{} Monomer cutoff for A2\_dimer-type descriptors

\item {} 
\sphinxstyleliteralstrong{\sphinxupquote{atomic\_number}} (\sphinxstyleliteralemphasis{\sphinxupquote{int}}) \textendash{} Atomic number in A2\_dimer-type descriptors

\end{itemize}

\end{description}\end{quote}

\end{fulllineitems}

\index{AB\_dimer (class in gap.descriptors)@\spxentry{AB\_dimer}\spxextra{class in gap.descriptors}}

\begin{fulllineitems}
\phantomsection\label{\detokenize{descriptors:gap.descriptors.AB_dimer}}\pysiglinewithargsret{\sphinxbfcode{\sphinxupquote{class }}\sphinxcode{\sphinxupquote{gap.descriptors.}}\sphinxbfcode{\sphinxupquote{AB\_dimer}}}{\emph{cutoff=0.0}, \emph{monomer\_cutoff=1.5}, \emph{atomic\_number1=1}, \emph{atomic\_number2=9}}{}~\begin{quote}\begin{description}
\item[{Parameters}] \leavevmode\begin{itemize}
\item {} 
\sphinxstyleliteralstrong{\sphinxupquote{cutoff}} (\sphinxstyleliteralemphasis{\sphinxupquote{float}}) \textendash{} Cutoff for AB\_dimer-type descriptors

\item {} 
\sphinxstyleliteralstrong{\sphinxupquote{monomer\_cutoff}} (\sphinxstyleliteralemphasis{\sphinxupquote{float}}) \textendash{} Monomer cutoff for AB\_dimer-type descriptors

\item {} 
\sphinxstyleliteralstrong{\sphinxupquote{atomic\_number1}} (\sphinxstyleliteralemphasis{\sphinxupquote{int}}) \textendash{} Atomic number of atom 1 in AB\_dimer-type descriptors

\item {} 
\sphinxstyleliteralstrong{\sphinxupquote{atomic\_number2}} (\sphinxstyleliteralemphasis{\sphinxupquote{int}}) \textendash{} Atomic number of atom 2 in AB\_dimer-type descriptors

\end{itemize}

\end{description}\end{quote}

\end{fulllineitems}

\index{AN\_monomer (class in gap.descriptors)@\spxentry{AN\_monomer}\spxextra{class in gap.descriptors}}

\begin{fulllineitems}
\phantomsection\label{\detokenize{descriptors:gap.descriptors.AN_monomer}}\pysiglinewithargsret{\sphinxbfcode{\sphinxupquote{class }}\sphinxcode{\sphinxupquote{gap.descriptors.}}\sphinxbfcode{\sphinxupquote{AN\_monomer}}}{\emph{cutoff=0.0}, \emph{atomic\_number=1}, \emph{N=4}, \emph{do\_atomic=True}}{}~\begin{quote}\begin{description}
\item[{Parameters}] \leavevmode\begin{itemize}
\item {} 
\sphinxstyleliteralstrong{\sphinxupquote{cutoff}} (\sphinxstyleliteralemphasis{\sphinxupquote{float}}) \textendash{} Cutoff for AN\_monomer-type descriptors

\item {} 
\sphinxstyleliteralstrong{\sphinxupquote{atomic\_number}} (\sphinxstyleliteralemphasis{\sphinxupquote{int}}) \textendash{} Atomic number in AN\_monomer-type descriptors

\item {} 
\sphinxstyleliteralstrong{\sphinxupquote{N}} (\sphinxstyleliteralemphasis{\sphinxupquote{int}}) \textendash{} Number of atoms in cluster

\item {} 
\sphinxstyleliteralstrong{\sphinxupquote{do\_atomic}} (\sphinxstyleliteralemphasis{\sphinxupquote{bool}}) \textendash{} Descriptors are cluster based or atom-based

\end{itemize}

\end{description}\end{quote}

\end{fulllineitems}

\index{DescriptorNew (class in gap.descriptors)@\spxentry{DescriptorNew}\spxextra{class in gap.descriptors}}

\begin{fulllineitems}
\phantomsection\label{\detokenize{descriptors:gap.descriptors.DescriptorNew}}\pysiglinewithargsret{\sphinxbfcode{\sphinxupquote{class }}\sphinxcode{\sphinxupquote{gap.descriptors.}}\sphinxbfcode{\sphinxupquote{DescriptorNew}}}{\emph{covariance\_type=None}, \emph{delta=None}, \emph{Name='Default'}, \emph{add\_species=False}, \emph{config\_type\_n\_sparse=None}, \emph{f0=0.0}, \emph{mark\_sparse\_atoms=False}, \emph{n\_sparse=0}, \emph{print\_sparse\_index=None}, \emph{sparse\_file=None}, \emph{sparse\_method=\textless{}sparse\_method.RANDOM: 'RANDOM'\textgreater{}}, \emph{theta\_fac=1.0}, \emph{theta\_file=None}, \emph{theta\_uniform=0.0}, \emph{unique\_descriptor\_tolerance=1e-10}, \emph{unique\_hash\_tolerance=1e-10}, \emph{zeta=1.0}}{}
GAP options
This subroutine parses the options given in the gap string, for each GAP.
\begin{quote}\begin{description}
\item[{Parameters}] \leavevmode\begin{itemize}
\item {} 
\sphinxstyleliteralstrong{\sphinxupquote{covariance\_type}} ({\hyperref[\detokenize{descriptors:gap.descriptors.covariance_type}]{\sphinxcrossref{\sphinxstyleliteralemphasis{\sphinxupquote{covariance\_type}}}}}) \textendash{} Type of covariance function to use. Available: ARD\_SE, DOT\_PRODUCT, BOND\_REAL\_SPACE, PP (piecewise polynomial)

\item {} 
\sphinxstyleliteralstrong{\sphinxupquote{delta}} (\sphinxstyleliteralemphasis{\sphinxupquote{float}}) \textendash{} Set the standard deviation of the Gaussian process. Typically this would be set to the standard deviation (i.e. root mean square) of the function that is approximated with the Gaussian process.

\item {} 
\sphinxstyleliteralstrong{\sphinxupquote{add\_species}} (\sphinxstyleliteralemphasis{\sphinxupquote{bool}}) \textendash{} Create species-specific descriptor, using the descriptor string as a template.

\item {} 
\sphinxstyleliteralstrong{\sphinxupquote{config\_type\_n\_sparse}} (\sphinxstyleliteralemphasis{\sphinxupquote{str}}) \textendash{} Number of sparse points in each config type. Format: \{type1:50:type2:100\}

\item {} 
\sphinxstyleliteralstrong{\sphinxupquote{f0}} (\sphinxstyleliteralemphasis{\sphinxupquote{float}}) \textendash{} Set the mean of the Gaussian process. Defaults to 0.

\item {} 
\sphinxstyleliteralstrong{\sphinxupquote{mark\_sparse\_atoms}} (\sphinxstyleliteralemphasis{\sphinxupquote{bool}}) \textendash{} Reprints the original xyz file after sparsification process. sparse propery added, true for atoms associated with a sparse point.

\item {} 
\sphinxstyleliteralstrong{\sphinxupquote{n\_sparse}} (\sphinxstyleliteralemphasis{\sphinxupquote{int}}) \textendash{} Number of sparse points to use in the sparsification of the Gaussian process

\item {} 
\sphinxstyleliteralstrong{\sphinxupquote{print\_sparse\_index}} (\sphinxstyleliteralemphasis{\sphinxupquote{str}}) \textendash{} If given, after determinining the sparse points, their 1-based indices are appended to this file

\item {} 
\sphinxstyleliteralstrong{\sphinxupquote{sparse\_file}} (\sphinxstyleliteralemphasis{\sphinxupquote{str}}) \textendash{} Sparse points from a file. Integers, in single line.

\item {} 
\sphinxstyleliteralstrong{\sphinxupquote{sparse\_method}} ({\hyperref[\detokenize{descriptors:gap.descriptors.sparse_method}]{\sphinxcrossref{\sphinxstyleliteralemphasis{\sphinxupquote{sparse\_method}}}}}) \textendash{} Sparsification method. RANDOM(default), PIVOT, CLUSTER, UNIFORM, KMEANS, COVARIANCE, NONE, FUZZY, FILE, INDEX\_FILE, CUR\_COVARIANCE, CUR\_POINTS

\item {} 
\sphinxstyleliteralstrong{\sphinxupquote{theta\_fac}} (\sphinxstyleliteralemphasis{\sphinxupquote{str}}) \textendash{} Set the width of Gaussians for the ARD\_SE and PP kernel by multiplying the range of each descriptor by theta\_fac. Can be a single number or different for each dimension. For multiple theta\_fac separate each value by whitespaces.

\item {} 
\sphinxstyleliteralstrong{\sphinxupquote{theta\_file}} (\sphinxstyleliteralemphasis{\sphinxupquote{str}}) \textendash{} Set the width of Gaussians for the ARD\_SE kernel from a file. There should be as many real numbers as the number of dimensions, in a single line

\item {} 
\sphinxstyleliteralstrong{\sphinxupquote{theta\_uniform}} (\sphinxstyleliteralemphasis{\sphinxupquote{float}}) \textendash{} Set the width of Gaussians for the ARD\_SE and PP kernel, same in each dimension.

\item {} 
\sphinxstyleliteralstrong{\sphinxupquote{unique\_descriptor\_tolerance}} (\sphinxstyleliteralemphasis{\sphinxupquote{float}}) \textendash{} Descriptor tolerance when filtering out duplicate data points

\item {} 
\sphinxstyleliteralstrong{\sphinxupquote{unique\_hash\_tolerance}} (\sphinxstyleliteralemphasis{\sphinxupquote{float}}) \textendash{} Hash tolerance when filtering out duplicate data points

\item {} 
\sphinxstyleliteralstrong{\sphinxupquote{zeta}} (\sphinxstyleliteralemphasis{\sphinxupquote{float}}) \textendash{} Exponent of soap type dot product covariance kernel

\end{itemize}

\end{description}\end{quote}

\end{fulllineitems}

\index{MissingParameter@\spxentry{MissingParameter}}

\begin{fulllineitems}
\phantomsection\label{\detokenize{descriptors:gap.descriptors.MissingParameter}}\pysiglinewithargsret{\sphinxbfcode{\sphinxupquote{exception }}\sphinxcode{\sphinxupquote{gap.descriptors.}}\sphinxbfcode{\sphinxupquote{MissingParameter}}}{\emph{parameter=None}}{}
\end{fulllineitems}

\index{alex (class in gap.descriptors)@\spxentry{alex}\spxextra{class in gap.descriptors}}

\begin{fulllineitems}
\phantomsection\label{\detokenize{descriptors:gap.descriptors.alex}}\pysiglinewithargsret{\sphinxbfcode{\sphinxupquote{class }}\sphinxcode{\sphinxupquote{gap.descriptors.}}\sphinxbfcode{\sphinxupquote{alex}}}{\emph{cutoff=0.0}, \emph{Z=0}, \emph{power\_min=5}, \emph{power\_max=10}, \emph{n\_species=1}, \emph{species\_Z=None}}{}~\begin{quote}\begin{description}
\item[{Parameters}] \leavevmode\begin{itemize}
\item {} 
\sphinxstyleliteralstrong{\sphinxupquote{cutoff}} (\sphinxstyleliteralemphasis{\sphinxupquote{float}}) \textendash{} Cutoff for alex-type descriptors

\item {} 
\sphinxstyleliteralstrong{\sphinxupquote{Z}} (\sphinxstyleliteralemphasis{\sphinxupquote{int}}) \textendash{} Atomic number of central atom

\item {} 
\sphinxstyleliteralstrong{\sphinxupquote{power\_min}} (\sphinxstyleliteralemphasis{\sphinxupquote{int}}) \textendash{} Minimum power of radial basis for the descriptor

\item {} 
\sphinxstyleliteralstrong{\sphinxupquote{power\_max}} (\sphinxstyleliteralemphasis{\sphinxupquote{int}}) \textendash{} Maximum power of the radial basis for the descriptor

\item {} 
\sphinxstyleliteralstrong{\sphinxupquote{n\_species}} (\sphinxstyleliteralemphasis{\sphinxupquote{int}}) \textendash{} Number of species for the descriptor

\item {} 
\sphinxstyleliteralstrong{\sphinxupquote{species\_Z}} \textendash{} Atomic number of species

\end{itemize}

\end{description}\end{quote}

\end{fulllineitems}

\index{angle\_3b (class in gap.descriptors)@\spxentry{angle\_3b}\spxextra{class in gap.descriptors}}

\begin{fulllineitems}
\phantomsection\label{\detokenize{descriptors:gap.descriptors.angle_3b}}\pysiglinewithargsret{\sphinxbfcode{\sphinxupquote{class }}\sphinxcode{\sphinxupquote{gap.descriptors.}}\sphinxbfcode{\sphinxupquote{angle\_3b}}}{\emph{cutoff=0.0}, \emph{Z=0}, \emph{Z1=0}, \emph{Z2=0}}{}~\begin{quote}\begin{description}
\item[{Parameters}] \leavevmode\begin{itemize}
\item {} 
\sphinxstyleliteralstrong{\sphinxupquote{cutoff}} (\sphinxstyleliteralemphasis{\sphinxupquote{float}}) \textendash{} Cutoff for angle\_3b-type descriptors

\item {} 
\sphinxstyleliteralstrong{\sphinxupquote{Z}} (\sphinxstyleliteralemphasis{\sphinxupquote{int}}) \textendash{} Atomic number of central atom

\item {} 
\sphinxstyleliteralstrong{\sphinxupquote{Z1}} (\sphinxstyleliteralemphasis{\sphinxupquote{int}}) \textendash{} Atomic number of neighbour \#1

\item {} 
\sphinxstyleliteralstrong{\sphinxupquote{Z2}} (\sphinxstyleliteralemphasis{\sphinxupquote{int}}) \textendash{} Atomic number of neighbour \#2

\end{itemize}

\end{description}\end{quote}

\end{fulllineitems}

\index{as\_distance\_2b (class in gap.descriptors)@\spxentry{as\_distance\_2b}\spxextra{class in gap.descriptors}}

\begin{fulllineitems}
\phantomsection\label{\detokenize{descriptors:gap.descriptors.as_distance_2b}}\pysiglinewithargsret{\sphinxbfcode{\sphinxupquote{class }}\sphinxcode{\sphinxupquote{gap.descriptors.}}\sphinxbfcode{\sphinxupquote{as\_distance\_2b}}}{\emph{min\_cutoff=0.0}, \emph{max\_cutoff=None}, \emph{as\_cutoff=None}, \emph{overlap\_alpha=0.5}, \emph{min\_transition\_width=0.5}, \emph{max\_transition\_width=0.5}, \emph{as\_transition\_width=0.1}, \emph{coordination\_cutoff=None}, \emph{coordination\_transition\_width=0.5}, \emph{Z1=0}, \emph{Z2=0}}{}~\begin{quote}\begin{description}
\item[{Parameters}] \leavevmode\begin{itemize}
\item {} 
\sphinxstyleliteralstrong{\sphinxupquote{min\_cutoff}} (\sphinxstyleliteralemphasis{\sphinxupquote{float}}) \textendash{} Lower cutoff for as\_distance\_2b-type descriptors

\item {} 
\sphinxstyleliteralstrong{\sphinxupquote{max\_cutoff}} \textendash{} Higher cutoff for as\_distance\_2b-type descriptors

\item {} 
\sphinxstyleliteralstrong{\sphinxupquote{as\_cutoff}} \textendash{} Cutoff of asymmetricity

\item {} 
\sphinxstyleliteralstrong{\sphinxupquote{overlap\_alpha}} (\sphinxstyleliteralemphasis{\sphinxupquote{float}}) \textendash{} Cutoff of asymmetricity

\item {} 
\sphinxstyleliteralstrong{\sphinxupquote{min\_transition\_width}} (\sphinxstyleliteralemphasis{\sphinxupquote{float}}) \textendash{} Transition width of lower cutoff for as\_distance\_2b-type descriptors

\item {} 
\sphinxstyleliteralstrong{\sphinxupquote{max\_transition\_width}} (\sphinxstyleliteralemphasis{\sphinxupquote{float}}) \textendash{} Transition width of higher cutoff for as\_distance\_2b-type descriptors

\item {} 
\sphinxstyleliteralstrong{\sphinxupquote{as\_transition\_width}} (\sphinxstyleliteralemphasis{\sphinxupquote{float}}) \textendash{} Transition width of asymmetricity cutoff for as\_distance\_2b-type descriptors

\item {} 
\sphinxstyleliteralstrong{\sphinxupquote{coordination\_cutoff}} \textendash{} Cutoff for coordination function in as\_distance\_2b-type descriptors

\item {} 
\sphinxstyleliteralstrong{\sphinxupquote{coordination\_transition\_width}} (\sphinxstyleliteralemphasis{\sphinxupquote{float}}) \textendash{} Transition width for as\_distance\_2b-type descriptors

\item {} 
\sphinxstyleliteralstrong{\sphinxupquote{Z1}} (\sphinxstyleliteralemphasis{\sphinxupquote{int}}) \textendash{} Atom type \#1 in bond

\item {} 
\sphinxstyleliteralstrong{\sphinxupquote{Z2}} (\sphinxstyleliteralemphasis{\sphinxupquote{int}}) \textendash{} Atom type \#2 in bond

\end{itemize}

\end{description}\end{quote}

\end{fulllineitems}

\index{atom\_real\_space (class in gap.descriptors)@\spxentry{atom\_real\_space}\spxextra{class in gap.descriptors}}

\begin{fulllineitems}
\phantomsection\label{\detokenize{descriptors:gap.descriptors.atom_real_space}}\pysiglinewithargsret{\sphinxbfcode{\sphinxupquote{class }}\sphinxcode{\sphinxupquote{gap.descriptors.}}\sphinxbfcode{\sphinxupquote{atom\_real\_space}}}{\emph{cutoff=0.0}, \emph{cutoff\_transition\_width=0.0}, \emph{l\_max=0}, \emph{alpha=1.0}, \emph{zeta=1.0}}{}~\begin{quote}\begin{description}
\item[{Parameters}] \leavevmode\begin{itemize}
\item {} 
\sphinxstyleliteralstrong{\sphinxupquote{cutoff}} (\sphinxstyleliteralemphasis{\sphinxupquote{float}}) \textendash{} Space cutoff for atom\_real\_space-type descriptors

\item {} 
\sphinxstyleliteralstrong{\sphinxupquote{cutoff\_transition\_width}} (\sphinxstyleliteralemphasis{\sphinxupquote{float}}) \textendash{} Space transition width for atom\_real\_space-type descriptors

\item {} 
\sphinxstyleliteralstrong{\sphinxupquote{l\_max}} (\sphinxstyleliteralemphasis{\sphinxupquote{int}}) \textendash{} Cutoff for spherical harmonics expansion

\item {} 
\sphinxstyleliteralstrong{\sphinxupquote{alpha}} (\sphinxstyleliteralemphasis{\sphinxupquote{float}}) \textendash{} Width of atomic Gaussians

\item {} 
\sphinxstyleliteralstrong{\sphinxupquote{zeta}} (\sphinxstyleliteralemphasis{\sphinxupquote{float}}) \textendash{} Exponent of covariance function

\end{itemize}

\end{description}\end{quote}

\end{fulllineitems}

\index{behler (class in gap.descriptors)@\spxentry{behler}\spxextra{class in gap.descriptors}}

\begin{fulllineitems}
\phantomsection\label{\detokenize{descriptors:gap.descriptors.behler}}\pysiglinewithargsret{\sphinxbfcode{\sphinxupquote{class }}\sphinxcode{\sphinxupquote{gap.descriptors.}}\sphinxbfcode{\sphinxupquote{behler}}}{\emph{behler\_cutoff=2.75}}{}~\begin{quote}\begin{description}
\item[{Parameters}] \leavevmode
\sphinxstyleliteralstrong{\sphinxupquote{behler\_cutoff}} (\sphinxstyleliteralemphasis{\sphinxupquote{float}}) \textendash{} Cutoff for Behler-type descriptors

\end{description}\end{quote}

\end{fulllineitems}

\index{bispectrum\_so3 (class in gap.descriptors)@\spxentry{bispectrum\_so3}\spxextra{class in gap.descriptors}}

\begin{fulllineitems}
\phantomsection\label{\detokenize{descriptors:gap.descriptors.bispectrum_so3}}\pysiglinewithargsret{\sphinxbfcode{\sphinxupquote{class }}\sphinxcode{\sphinxupquote{gap.descriptors.}}\sphinxbfcode{\sphinxupquote{bispectrum\_so3}}}{\emph{cutoff=0.0}, \emph{min\_cutoff=0.0}, \emph{l\_max=4}, \emph{n\_max=4}, \emph{Z=0}, \emph{n\_species=1}, \emph{species\_Z=None}, \emph{w=None}}{}~\begin{quote}\begin{description}
\item[{Parameters}] \leavevmode\begin{itemize}
\item {} 
\sphinxstyleliteralstrong{\sphinxupquote{cutoff}} (\sphinxstyleliteralemphasis{\sphinxupquote{float}}) \textendash{} Cutoff for bispectrum\_so3-type descriptors

\item {} 
\sphinxstyleliteralstrong{\sphinxupquote{min\_cutoff}} (\sphinxstyleliteralemphasis{\sphinxupquote{float}}) \textendash{} Cutoff for minimal distances in bispectrum\_so3-type descriptors

\item {} 
\sphinxstyleliteralstrong{\sphinxupquote{l\_max}} (\sphinxstyleliteralemphasis{\sphinxupquote{int}}) \textendash{} L\_max for bispectrum\_so3-type descriptors

\item {} 
\sphinxstyleliteralstrong{\sphinxupquote{n\_max}} (\sphinxstyleliteralemphasis{\sphinxupquote{int}}) \textendash{} N\_max for bispectrum\_so3-type descriptors

\item {} 
\sphinxstyleliteralstrong{\sphinxupquote{Z}} (\sphinxstyleliteralemphasis{\sphinxupquote{int}}) \textendash{} Atomic number of central atom

\item {} 
\sphinxstyleliteralstrong{\sphinxupquote{n\_species}} (\sphinxstyleliteralemphasis{\sphinxupquote{int}}) \textendash{} Number of species for the descriptor

\item {} 
\sphinxstyleliteralstrong{\sphinxupquote{species\_Z}} (\sphinxstyleliteralemphasis{\sphinxupquote{int \_\_OR\_\_ ?}}\sphinxstyleliteralemphasis{\sphinxupquote{)}}) \textendash{} Atomic number of species

\item {} 
\sphinxstyleliteralstrong{\sphinxupquote{w}} (\sphinxstyleliteralemphasis{\sphinxupquote{float \_\_OR\_\_ ?}}\sphinxstyleliteralemphasis{\sphinxupquote{)}}) \textendash{} Weight associated to each atomic type

\end{itemize}

\end{description}\end{quote}

\end{fulllineitems}

\index{bispectrum\_so4 (class in gap.descriptors)@\spxentry{bispectrum\_so4}\spxextra{class in gap.descriptors}}

\begin{fulllineitems}
\phantomsection\label{\detokenize{descriptors:gap.descriptors.bispectrum_so4}}\pysigline{\sphinxbfcode{\sphinxupquote{class }}\sphinxcode{\sphinxupquote{gap.descriptors.}}\sphinxbfcode{\sphinxupquote{bispectrum\_so4}}}
Args:

\end{fulllineitems}

\index{bond\_real\_space (class in gap.descriptors)@\spxentry{bond\_real\_space}\spxextra{class in gap.descriptors}}

\begin{fulllineitems}
\phantomsection\label{\detokenize{descriptors:gap.descriptors.bond_real_space}}\pysiglinewithargsret{\sphinxbfcode{\sphinxupquote{class }}\sphinxcode{\sphinxupquote{gap.descriptors.}}\sphinxbfcode{\sphinxupquote{bond\_real\_space}}}{\emph{bond\_cutoff=0.0}, \emph{bond\_transition\_width=0.0}, \emph{cutoff=0.0}, \emph{transition\_width=0.0}, \emph{atom\_sigma=0.0}, \emph{max\_neighbours=0}}{}~\begin{quote}\begin{description}
\item[{Parameters}] \leavevmode\begin{itemize}
\item {} 
\sphinxstyleliteralstrong{\sphinxupquote{bond\_cutoff}} (\sphinxstyleliteralemphasis{\sphinxupquote{float}}) \textendash{} Bond cutoff for bond\_real\_space-type descriptors

\item {} 
\sphinxstyleliteralstrong{\sphinxupquote{bond\_transition\_width}} (\sphinxstyleliteralemphasis{\sphinxupquote{float}}) \textendash{} Bond transition width for bond\_real\_space-type descriptors

\item {} 
\sphinxstyleliteralstrong{\sphinxupquote{cutoff}} (\sphinxstyleliteralemphasis{\sphinxupquote{float}}) \textendash{} Space cutoff for bond\_real\_space-type descriptors

\item {} 
\sphinxstyleliteralstrong{\sphinxupquote{transition\_width}} (\sphinxstyleliteralemphasis{\sphinxupquote{float}}) \textendash{} Space transition width for bond\_real\_space-type descriptors

\item {} 
\sphinxstyleliteralstrong{\sphinxupquote{atom\_sigma}} (\sphinxstyleliteralemphasis{\sphinxupquote{float}}) \textendash{} Atom sigma for bond\_real\_space-type descriptors

\item {} 
\sphinxstyleliteralstrong{\sphinxupquote{max\_neighbours}} (\sphinxstyleliteralemphasis{\sphinxupquote{int}}) \textendash{} Maximum number of neighbours

\end{itemize}

\end{description}\end{quote}

\end{fulllineitems}

\index{co\_angle\_3b (class in gap.descriptors)@\spxentry{co\_angle\_3b}\spxextra{class in gap.descriptors}}

\begin{fulllineitems}
\phantomsection\label{\detokenize{descriptors:gap.descriptors.co_angle_3b}}\pysiglinewithargsret{\sphinxbfcode{\sphinxupquote{class }}\sphinxcode{\sphinxupquote{gap.descriptors.}}\sphinxbfcode{\sphinxupquote{co\_angle\_3b}}}{\emph{cutoff=0.0}, \emph{coordination\_cutoff=0.0}, \emph{coordination\_transition\_width=0.0}, \emph{Z=0}, \emph{Z1=0}, \emph{Z2=0}}{}~\begin{quote}\begin{description}
\item[{Parameters}] \leavevmode\begin{itemize}
\item {} 
\sphinxstyleliteralstrong{\sphinxupquote{cutoff}} (\sphinxstyleliteralemphasis{\sphinxupquote{float}}) \textendash{} Cutoff for co\_angle\_3b-type descriptors

\item {} 
\sphinxstyleliteralstrong{\sphinxupquote{coordination\_cutoff}} (\sphinxstyleliteralemphasis{\sphinxupquote{float}}) \textendash{} Cutoff for coordination function in co\_angle\_3b-type descriptors

\item {} 
\sphinxstyleliteralstrong{\sphinxupquote{coordination\_transition\_width}} (\sphinxstyleliteralemphasis{\sphinxupquote{float}}) \textendash{} Transition width for co\_angle\_3b-type descriptors

\item {} 
\sphinxstyleliteralstrong{\sphinxupquote{Z}} (\sphinxstyleliteralemphasis{\sphinxupquote{int}}) \textendash{} Atomic number of central atom

\item {} 
\sphinxstyleliteralstrong{\sphinxupquote{Z1}} (\sphinxstyleliteralemphasis{\sphinxupquote{int}}) \textendash{} Atomic number of neighbour \#1

\item {} 
\sphinxstyleliteralstrong{\sphinxupquote{Z2}} (\sphinxstyleliteralemphasis{\sphinxupquote{int}}) \textendash{} Atomic number of neighbour \#2

\end{itemize}

\end{description}\end{quote}

\end{fulllineitems}

\index{co\_distance\_2b (class in gap.descriptors)@\spxentry{co\_distance\_2b}\spxextra{class in gap.descriptors}}

\begin{fulllineitems}
\phantomsection\label{\detokenize{descriptors:gap.descriptors.co_distance_2b}}\pysiglinewithargsret{\sphinxbfcode{\sphinxupquote{class }}\sphinxcode{\sphinxupquote{gap.descriptors.}}\sphinxbfcode{\sphinxupquote{co\_distance\_2b}}}{\emph{cutoff=0.0}, \emph{transition\_width=0.5}, \emph{coordination\_cutoff=0.0}, \emph{coordination\_transition\_width=0.0}, \emph{Z1=0}, \emph{Z2=0}}{}~\begin{quote}\begin{description}
\item[{Parameters}] \leavevmode\begin{itemize}
\item {} 
\sphinxstyleliteralstrong{\sphinxupquote{cutoff}} (\sphinxstyleliteralemphasis{\sphinxupquote{float}}) \textendash{} Cutoff for co\_distance\_2b-type descriptors

\item {} 
\sphinxstyleliteralstrong{\sphinxupquote{transition\_width}} (\sphinxstyleliteralemphasis{\sphinxupquote{float}}) \textendash{} Transition width of cutoff for co\_distance\_2b-type descriptors

\item {} 
\sphinxstyleliteralstrong{\sphinxupquote{coordination\_cutoff}} (\sphinxstyleliteralemphasis{\sphinxupquote{float}}) \textendash{} Cutoff for coordination function in co\_distance\_2b-type descriptors

\item {} 
\sphinxstyleliteralstrong{\sphinxupquote{coordination\_transition\_width}} (\sphinxstyleliteralemphasis{\sphinxupquote{float}}) \textendash{} Transition width for co\_distance\_2b-type descriptors

\item {} 
\sphinxstyleliteralstrong{\sphinxupquote{Z1}} (\sphinxstyleliteralemphasis{\sphinxupquote{int}}) \textendash{} Atom type \#1 in bond

\item {} 
\sphinxstyleliteralstrong{\sphinxupquote{Z2}} (\sphinxstyleliteralemphasis{\sphinxupquote{int}}) \textendash{} Atom type \#2 in bond

\end{itemize}

\end{description}\end{quote}

\end{fulllineitems}

\index{com\_dimer (class in gap.descriptors)@\spxentry{com\_dimer}\spxextra{class in gap.descriptors}}

\begin{fulllineitems}
\phantomsection\label{\detokenize{descriptors:gap.descriptors.com_dimer}}\pysiglinewithargsret{\sphinxbfcode{\sphinxupquote{class }}\sphinxcode{\sphinxupquote{gap.descriptors.}}\sphinxbfcode{\sphinxupquote{com\_dimer}}}{\emph{cutoff=0.0}, \emph{monomer\_one\_cutoff=0.0}, \emph{monomer\_two\_cutoff=0.0}, \emph{cutoff\_transition\_width=0.5}, \emph{atom\_ordercheck=True}, \emph{strict=True}, \emph{mpifind=False}, \emph{signature\_one=None}, \emph{signature\_two=None}}{}~\begin{quote}\begin{description}
\item[{Parameters}] \leavevmode\begin{itemize}
\item {} 
\sphinxstyleliteralstrong{\sphinxupquote{cutoff}} (\sphinxstyleliteralemphasis{\sphinxupquote{float}}) \textendash{} Cutoff(intermolecular) for com\_dimer-type descriptors

\item {} 
\sphinxstyleliteralstrong{\sphinxupquote{monomer\_one\_cutoff}} (\sphinxstyleliteralemphasis{\sphinxupquote{float}}) \textendash{} Cutoff(mono1) for com\_dimer-type descriptors

\item {} 
\sphinxstyleliteralstrong{\sphinxupquote{monomer\_two\_cutoff}} (\sphinxstyleliteralemphasis{\sphinxupquote{float}}) \textendash{} Cutoff(mono2) for com\_dimer-type descriptors

\item {} 
\sphinxstyleliteralstrong{\sphinxupquote{cutoff\_transition\_width}} (\sphinxstyleliteralemphasis{\sphinxupquote{float}}) \textendash{} Width of smooth cutoff region for com\_dimer-type descriptors

\item {} 
\sphinxstyleliteralstrong{\sphinxupquote{atom\_ordercheck}} (\sphinxstyleliteralemphasis{\sphinxupquote{bool}}) \textendash{} T: find molecules. F: go by order of atoms

\item {} 
\sphinxstyleliteralstrong{\sphinxupquote{strict}} (\sphinxstyleliteralemphasis{\sphinxupquote{bool}}) \textendash{} Raise error if not all atoms assigned to monomer or if no monomer pairs found

\item {} 
\sphinxstyleliteralstrong{\sphinxupquote{mpifind}} (\sphinxstyleliteralemphasis{\sphinxupquote{bool}}) \textendash{} Use find\_monomer\_pairs\_MPI

\item {} 
\sphinxstyleliteralstrong{\sphinxupquote{signature\_one}} \textendash{} Atomic numbers of monomer one, format \{Z1 Z2 Z3 …\}

\item {} 
\sphinxstyleliteralstrong{\sphinxupquote{signature\_two}} \textendash{} Atomic numbers of monomer two, format \{Z1 Z2 Z3 …\}

\end{itemize}

\end{description}\end{quote}

\end{fulllineitems}

\index{coordination (class in gap.descriptors)@\spxentry{coordination}\spxextra{class in gap.descriptors}}

\begin{fulllineitems}
\phantomsection\label{\detokenize{descriptors:gap.descriptors.coordination}}\pysiglinewithargsret{\sphinxbfcode{\sphinxupquote{class }}\sphinxcode{\sphinxupquote{gap.descriptors.}}\sphinxbfcode{\sphinxupquote{coordination}}}{\emph{cutoff=0.0}, \emph{transition\_width=0.2}, \emph{Z=0}}{}~\begin{quote}\begin{description}
\item[{Parameters}] \leavevmode\begin{itemize}
\item {} 
\sphinxstyleliteralstrong{\sphinxupquote{cutoff}} (\sphinxstyleliteralemphasis{\sphinxupquote{float}}) \textendash{} Cutoff for coordination-type descriptors

\item {} 
\sphinxstyleliteralstrong{\sphinxupquote{transition\_width}} (\sphinxstyleliteralemphasis{\sphinxupquote{float}}) \textendash{} Width of transition region from 1 to 0

\item {} 
\sphinxstyleliteralstrong{\sphinxupquote{Z}} (\sphinxstyleliteralemphasis{\sphinxupquote{int}}) \textendash{} Atomic number of central atom

\end{itemize}

\end{description}\end{quote}

\end{fulllineitems}

\index{cosnx (class in gap.descriptors)@\spxentry{cosnx}\spxextra{class in gap.descriptors}}

\begin{fulllineitems}
\phantomsection\label{\detokenize{descriptors:gap.descriptors.cosnx}}\pysiglinewithargsret{\sphinxbfcode{\sphinxupquote{class }}\sphinxcode{\sphinxupquote{gap.descriptors.}}\sphinxbfcode{\sphinxupquote{cosnx}}}{\emph{cutoff=0.0}, \emph{min\_cutoff=0.0}, \emph{l\_max=4}, \emph{n\_max=4}, \emph{Z=0}, \emph{n\_species=1}, \emph{species\_Z=None}, \emph{w=None}}{}~\begin{quote}\begin{description}
\item[{Parameters}] \leavevmode\begin{itemize}
\item {} 
\sphinxstyleliteralstrong{\sphinxupquote{cutoff}} (\sphinxstyleliteralemphasis{\sphinxupquote{float}}) \textendash{} Cutoff for cosnx-type descriptors

\item {} 
\sphinxstyleliteralstrong{\sphinxupquote{min\_cutoff}} (\sphinxstyleliteralemphasis{\sphinxupquote{float}}) \textendash{} Cutoff for minimal distances in cosnx-type descriptors

\item {} 
\sphinxstyleliteralstrong{\sphinxupquote{l\_max}} (\sphinxstyleliteralemphasis{\sphinxupquote{int}}) \textendash{} L\_max for cosnx-type descriptors

\item {} 
\sphinxstyleliteralstrong{\sphinxupquote{n\_max}} (\sphinxstyleliteralemphasis{\sphinxupquote{int}}) \textendash{} N\_max for cosnx-type descriptors

\item {} 
\sphinxstyleliteralstrong{\sphinxupquote{Z}} (\sphinxstyleliteralemphasis{\sphinxupquote{int}}) \textendash{} Atomic number of central atom

\item {} 
\sphinxstyleliteralstrong{\sphinxupquote{n\_species}} (\sphinxstyleliteralemphasis{\sphinxupquote{int}}) \textendash{} Number of species for the descriptor

\item {} 
\sphinxstyleliteralstrong{\sphinxupquote{species\_Z}} (\sphinxstyleliteralemphasis{\sphinxupquote{int \_\_OR\_\_ ?}}\sphinxstyleliteralemphasis{\sphinxupquote{)}}) \textendash{} Atomic number of species

\item {} 
\sphinxstyleliteralstrong{\sphinxupquote{w}} (\sphinxstyleliteralemphasis{\sphinxupquote{float \_\_OR\_\_ ?}}\sphinxstyleliteralemphasis{\sphinxupquote{)}}) \textendash{} Weight associated to each atomic type

\end{itemize}

\end{description}\end{quote}

\end{fulllineitems}

\index{covariance\_type (class in gap.descriptors)@\spxentry{covariance\_type}\spxextra{class in gap.descriptors}}

\begin{fulllineitems}
\phantomsection\label{\detokenize{descriptors:gap.descriptors.covariance_type}}\pysigline{\sphinxbfcode{\sphinxupquote{class }}\sphinxcode{\sphinxupquote{gap.descriptors.}}\sphinxbfcode{\sphinxupquote{covariance\_type}}}
Type of covariance function to use. Available:
\begin{itemize}
\item {} 
\sphinxstylestrong{ARD\_SE},

\item {} 
\sphinxstylestrong{DOT\_PRODUCT},

\item {} 
\sphinxstylestrong{BOND\_REAL\_SPACE},

\item {} 
\sphinxstylestrong{PP}: piecewise polynomial

\end{itemize}

\end{fulllineitems}

\index{distance\_2b (class in gap.descriptors)@\spxentry{distance\_2b}\spxextra{class in gap.descriptors}}

\begin{fulllineitems}
\phantomsection\label{\detokenize{descriptors:gap.descriptors.distance_2b}}\pysiglinewithargsret{\sphinxbfcode{\sphinxupquote{class }}\sphinxcode{\sphinxupquote{gap.descriptors.}}\sphinxbfcode{\sphinxupquote{distance\_2b}}}{\emph{cutoff=0.0}, \emph{cutoff\_transition\_width=0.5}, \emph{Z1=0}, \emph{Z2=0}, \emph{resid\_name=''}, \emph{only\_intra=False}, \emph{only\_inter=False}, \emph{**kwargs}}{}~\begin{quote}\begin{description}
\item[{Parameters}] \leavevmode\begin{itemize}
\item {} 
\sphinxstyleliteralstrong{\sphinxupquote{cutoff}} (\sphinxstyleliteralemphasis{\sphinxupquote{float}}) \textendash{} Cutoff for distance\_2b-type descriptors

\item {} 
\sphinxstyleliteralstrong{\sphinxupquote{cutoff\_transition\_width}} (\sphinxstyleliteralemphasis{\sphinxupquote{float}}) \textendash{} Transition width of cutoff for distance\_2b-type descriptors

\item {} 
\sphinxstyleliteralstrong{\sphinxupquote{Z1}} (\sphinxstyleliteralemphasis{\sphinxupquote{int}}) \textendash{} Atom type \#1 in bond

\item {} 
\sphinxstyleliteralstrong{\sphinxupquote{Z2}} (\sphinxstyleliteralemphasis{\sphinxupquote{int}}) \textendash{} Atom type \#2 in bond

\item {} 
\sphinxstyleliteralstrong{\sphinxupquote{resid\_name}} (\sphinxstyleliteralemphasis{\sphinxupquote{str}}) \textendash{} Name of an integer property in the atoms object giving the residue id of the molecule to which the atom belongs.

\item {} 
\sphinxstyleliteralstrong{\sphinxupquote{only\_intra}} (\sphinxstyleliteralemphasis{\sphinxupquote{bool}}) \textendash{} Only calculate INTRAmolecular pairs with equal residue ids (bonds)

\item {} 
\sphinxstyleliteralstrong{\sphinxupquote{only\_inter}} (\sphinxstyleliteralemphasis{\sphinxupquote{bool}}) \textendash{} Only apply to INTERmolecular pairs with different residue ids (non-bonded)

\end{itemize}

\end{description}\end{quote}

\end{fulllineitems}

\index{distance\_Nb (class in gap.descriptors)@\spxentry{distance\_Nb}\spxextra{class in gap.descriptors}}

\begin{fulllineitems}
\phantomsection\label{\detokenize{descriptors:gap.descriptors.distance_Nb}}\pysiglinewithargsret{\sphinxbfcode{\sphinxupquote{class }}\sphinxcode{\sphinxupquote{gap.descriptors.}}\sphinxbfcode{\sphinxupquote{distance\_Nb}}}{\emph{order}, \emph{cutoff}, \emph{cutoff\_transition\_width=0.5}, \emph{compact\_clusters=False}, \emph{Z=None}, \emph{atom\_mask\_name=None}, \emph{xml\_version=1423143769}, \emph{do\_transfer=False}, \emph{transfer\_factor=5.0}, \emph{transfer\_width=1.0}, \emph{transfer\_r0=3.0}}{}~\begin{quote}\begin{description}
\item[{Parameters}] \leavevmode\begin{itemize}
\item {} 
\sphinxstyleliteralstrong{\sphinxupquote{cutoff}} \textendash{} Cutoff for distance\_Nb-type descriptors

\item {} 
\sphinxstyleliteralstrong{\sphinxupquote{cutoff\_transition\_width}} (\sphinxstyleliteralemphasis{\sphinxupquote{float}}) \textendash{} Transition width of cutoff for distance\_Nb-type descriptors

\item {} 
\sphinxstyleliteralstrong{\sphinxupquote{order}} \textendash{} Many-body order, in terms of number of neighbours

\item {} 
\sphinxstyleliteralstrong{\sphinxupquote{compact\_clusters}} (\sphinxstyleliteralemphasis{\sphinxupquote{bool}}) \textendash{} If true, generate clusters where the atoms have at least one connection to the central atom. If false, only clusters where all atoms are connected are generated.

\item {} 
\sphinxstyleliteralstrong{\sphinxupquote{Z}} \textendash{} Atomic type of neighbours

\item {} 
\sphinxstyleliteralstrong{\sphinxupquote{atom\_mask\_name=None}} \textendash{} Name of a logical property in the atoms object. For atoms where this property is true descriptors are calculated.

\item {} 
\sphinxstyleliteralstrong{\sphinxupquote{xml\_version}} (\sphinxstyleliteralemphasis{\sphinxupquote{int}}) \textendash{} Version of GAP the XML potential file was created

\item {} 
\sphinxstyleliteralstrong{\sphinxupquote{do\_transfer}} (\sphinxstyleliteralemphasis{\sphinxupquote{bool}}) \textendash{} Enable transfer function

\item {} 
\sphinxstyleliteralstrong{\sphinxupquote{transfer\_factor}} (\sphinxstyleliteralemphasis{\sphinxupquote{float}}) \textendash{} Transfer function: stretch factor

\item {} 
\sphinxstyleliteralstrong{\sphinxupquote{transfer\_width}} (\sphinxstyleliteralemphasis{\sphinxupquote{float}}) \textendash{} Transfer function: transition width

\item {} 
\sphinxstyleliteralstrong{\sphinxupquote{transfer\_r0}} (\sphinxstyleliteralemphasis{\sphinxupquote{float}}) \textendash{} Transfer function: transition distance

\end{itemize}

\end{description}\end{quote}

\end{fulllineitems}

\index{fourier\_so4 (class in gap.descriptors)@\spxentry{fourier\_so4}\spxextra{class in gap.descriptors}}

\begin{fulllineitems}
\phantomsection\label{\detokenize{descriptors:gap.descriptors.fourier_so4}}\pysiglinewithargsret{\sphinxbfcode{\sphinxupquote{class }}\sphinxcode{\sphinxupquote{gap.descriptors.}}\sphinxbfcode{\sphinxupquote{fourier\_so4}}}{\emph{cutoff=2.75}, \emph{z0\_ratio=0.0}, \emph{j\_max=4}, \emph{Z=0}, \emph{n\_species=1}, \emph{species\_Z=None}, \emph{w=None}}{}~\begin{quote}\begin{description}
\item[{Parameters}] \leavevmode\begin{itemize}
\item {} 
\sphinxstyleliteralstrong{\sphinxupquote{cutoff}} (\sphinxstyleliteralemphasis{\sphinxupquote{float}}) \textendash{} Cutoff for SO4 bispectrum

\item {} 
\sphinxstyleliteralstrong{\sphinxupquote{z0\_ratio}} (\sphinxstyleliteralemphasis{\sphinxupquote{float}}) \textendash{} Ratio of radius of 4D projection sphere times PI and the cutoff.

\item {} 
\sphinxstyleliteralstrong{\sphinxupquote{j\_max}} (\sphinxstyleliteralemphasis{\sphinxupquote{int}}) \textendash{} Max of expansion of bispectrum, i.e. resulution

\item {} 
\sphinxstyleliteralstrong{\sphinxupquote{Z}} (\sphinxstyleliteralemphasis{\sphinxupquote{int}}) \textendash{} Atomic number of central atom

\item {} 
\sphinxstyleliteralstrong{\sphinxupquote{n\_species}} (\sphinxstyleliteralemphasis{\sphinxupquote{int}}) \textendash{} Number of species for the descriptor

\item {} 
\sphinxstyleliteralstrong{\sphinxupquote{species\_Z}} (\sphinxstyleliteralemphasis{\sphinxupquote{int \_\_OR\_\_ ?}}) \textendash{} Atomic number of species

\item {} 
\sphinxstyleliteralstrong{\sphinxupquote{w}} (\sphinxstyleliteralemphasis{\sphinxupquote{float \_\_OR\_\_ ?}}) \textendash{} Weight associated to each atomic type

\end{itemize}

\end{description}\end{quote}

\end{fulllineitems}

\index{general\_dimer (class in gap.descriptors)@\spxentry{general\_dimer}\spxextra{class in gap.descriptors}}

\begin{fulllineitems}
\phantomsection\label{\detokenize{descriptors:gap.descriptors.general_dimer}}\pysiglinewithargsret{\sphinxbfcode{\sphinxupquote{class }}\sphinxcode{\sphinxupquote{gap.descriptors.}}\sphinxbfcode{\sphinxupquote{general\_dimer}}}{\emph{cutoff=0.0}, \emph{monomer\_one\_cutoff=0.0}, \emph{monomer\_two\_cutoff=0.0}, \emph{cutoff\_transition\_width=0.5}, \emph{internal\_swaps\_only=True}, \emph{atom\_ordercheck=True}, \emph{double\_count=False}, \emph{strict=True}, \emph{use\_com=False}, \emph{mpifind=False}, \emph{signature\_one=None}, \emph{signature\_two=None}}{}~\begin{quote}\begin{description}
\item[{Parameters}] \leavevmode\begin{itemize}
\item {} 
\sphinxstyleliteralstrong{\sphinxupquote{cutoff}} (\sphinxstyleliteralemphasis{\sphinxupquote{float}}) \textendash{} Cutoff(intermolecular) for general\_dimer-type descriptors

\item {} 
\sphinxstyleliteralstrong{\sphinxupquote{monomer\_one\_cutoff}} (\sphinxstyleliteralemphasis{\sphinxupquote{float}}) \textendash{} Cutoff(mono1) for general\_dimer-type descriptors

\item {} 
\sphinxstyleliteralstrong{\sphinxupquote{monomer\_two\_cutoff}} (\sphinxstyleliteralemphasis{\sphinxupquote{float}}) \textendash{} Cutoff(mono2) for general\_dimer-type descriptors

\item {} 
\sphinxstyleliteralstrong{\sphinxupquote{cutoff\_transition\_width}} (\sphinxstyleliteralemphasis{\sphinxupquote{float}}) \textendash{} Width of smooth cutoff region for general\_dimer-type descriptors

\item {} 
\sphinxstyleliteralstrong{\sphinxupquote{internal\_swaps\_only}} (\sphinxstyleliteralemphasis{\sphinxupquote{bool}}) \textendash{} F: energies will be symmetrised over swaps of nuclei between monomers

\item {} 
\sphinxstyleliteralstrong{\sphinxupquote{atom\_ordercheck}} (\sphinxstyleliteralemphasis{\sphinxupquote{bool}}) \textendash{} T: find molecules. F: go by order of atoms

\item {} 
\sphinxstyleliteralstrong{\sphinxupquote{double\_count}} (\sphinxstyleliteralemphasis{\sphinxupquote{bool}}) \textendash{} T: double count when constructing the dimers, for compatibility with water dimer descriptor, default False

\item {} 
\sphinxstyleliteralstrong{\sphinxupquote{strict}} (\sphinxstyleliteralemphasis{\sphinxupquote{bool}}) \textendash{} Raise error if not all atoms assigned to monomer or if no monomer pairs found

\item {} 
\sphinxstyleliteralstrong{\sphinxupquote{use\_com}} (\sphinxstyleliteralemphasis{\sphinxupquote{bool}}) \textendash{} Use COM instead of COG

\item {} 
\sphinxstyleliteralstrong{\sphinxupquote{mpifind}} (\sphinxstyleliteralemphasis{\sphinxupquote{bool}}) \textendash{} Use find\_monomer\_pairs\_MPI

\item {} 
\sphinxstyleliteralstrong{\sphinxupquote{signature\_one}} \textendash{} Atomic numbers of monomer one, format \{Z1 Z2 Z3 …\}

\item {} 
\sphinxstyleliteralstrong{\sphinxupquote{signature\_two}} \textendash{} Atomic numbers of monomer two, format \{Z1 Z2 Z3 …\}

\end{itemize}

\end{description}\end{quote}

\end{fulllineitems}

\index{general\_monomer (class in gap.descriptors)@\spxentry{general\_monomer}\spxextra{class in gap.descriptors}}

\begin{fulllineitems}
\phantomsection\label{\detokenize{descriptors:gap.descriptors.general_monomer}}\pysiglinewithargsret{\sphinxbfcode{\sphinxupquote{class }}\sphinxcode{\sphinxupquote{gap.descriptors.}}\sphinxbfcode{\sphinxupquote{general\_monomer}}}{\emph{cutoff=0.0}, \emph{signature=None}, \emph{atom\_ordercheck=True}, \emph{strict=True}, \emph{power=1.0}}{}~\begin{quote}\begin{description}
\item[{Parameters}] \leavevmode\begin{itemize}
\item {} 
\sphinxstyleliteralstrong{\sphinxupquote{cutoff}} (\sphinxstyleliteralemphasis{\sphinxupquote{float}}) \textendash{} Cutoff for general\_monomer-type descriptors

\item {} 
\sphinxstyleliteralstrong{\sphinxupquote{signature}} \textendash{} Atomic numbers of monomer one, format \{Z1 Z2 Z3 …\}

\item {} 
\sphinxstyleliteralstrong{\sphinxupquote{atom\_ordercheck}} (\sphinxstyleliteralemphasis{\sphinxupquote{bool}}) \textendash{} T: find molecules. F: go by order of atoms

\item {} 
\sphinxstyleliteralstrong{\sphinxupquote{strict}} (\sphinxstyleliteralemphasis{\sphinxupquote{bool}}) \textendash{} Raise error if not all atoms assigned to monomer

\item {} 
\sphinxstyleliteralstrong{\sphinxupquote{power}} (\sphinxstyleliteralemphasis{\sphinxupquote{float}}) \textendash{} Power of distances to be used in the kernel

\end{itemize}

\end{description}\end{quote}

\end{fulllineitems}

\index{general\_trimer (class in gap.descriptors)@\spxentry{general\_trimer}\spxextra{class in gap.descriptors}}

\begin{fulllineitems}
\phantomsection\label{\detokenize{descriptors:gap.descriptors.general_trimer}}\pysiglinewithargsret{\sphinxbfcode{\sphinxupquote{class }}\sphinxcode{\sphinxupquote{gap.descriptors.}}\sphinxbfcode{\sphinxupquote{general\_trimer}}}{\emph{cutoff=0.0}, \emph{monomer\_one\_cutoff=0.0}, \emph{monomer\_two\_cutoff=0.0}, \emph{monomer\_three\_cutoff=0.0}, \emph{cutoff\_transition\_width=0.5}, \emph{internal\_swaps\_only=True}, \emph{atom\_ordercheck=True}, \emph{strict=True}, \emph{use\_com=False}, \emph{mpifind=False}, \emph{signature\_one=None}, \emph{signature\_two=None}, \emph{signature\_three=None}, \emph{power=1.0}}{}~\begin{quote}\begin{description}
\item[{Parameters}] \leavevmode\begin{itemize}
\item {} 
\sphinxstyleliteralstrong{\sphinxupquote{cutoff}} (\sphinxstyleliteralemphasis{\sphinxupquote{float}}) \textendash{} Cutoff(intermolecular) for general\_trimer-type descriptors

\item {} 
\sphinxstyleliteralstrong{\sphinxupquote{monomer\_one\_cutoff}} (\sphinxstyleliteralemphasis{\sphinxupquote{float}}) \textendash{} Cutoff(mono1) for general\_trimer-type descriptors

\item {} 
\sphinxstyleliteralstrong{\sphinxupquote{monomer\_two\_cutoff}} (\sphinxstyleliteralemphasis{\sphinxupquote{float}}) \textendash{} Cutoff(mono2) for general\_trimer-type descriptors

\item {} 
\sphinxstyleliteralstrong{\sphinxupquote{monomer\_three\_cutoff}} (\sphinxstyleliteralemphasis{\sphinxupquote{float}}) \textendash{} Cutoff(mono3) for general\_trimer-type descriptors

\item {} 
\sphinxstyleliteralstrong{\sphinxupquote{cutoff\_transition\_width}} (\sphinxstyleliteralemphasis{\sphinxupquote{float}}) \textendash{} Width of smooth cutoff region for general\_trimer-type descriptors

\item {} 
\sphinxstyleliteralstrong{\sphinxupquote{internal\_swaps\_only}} (\sphinxstyleliteralemphasis{\sphinxupquote{bool}}) \textendash{} F: energies will be symmetrised over swaps of nuclei between monomers

\item {} 
\sphinxstyleliteralstrong{\sphinxupquote{atom\_ordercheck}} (\sphinxstyleliteralemphasis{\sphinxupquote{bool}}) \textendash{} T: find molecules. F: go by order of atoms

\item {} 
\sphinxstyleliteralstrong{\sphinxupquote{strict}} (\sphinxstyleliteralemphasis{\sphinxupquote{bool}}) \textendash{} Raise error if not all atoms assigned to monomer or if no monomer pairs found

\item {} 
\sphinxstyleliteralstrong{\sphinxupquote{use\_com}} (\sphinxstyleliteralemphasis{\sphinxupquote{bool}}) \textendash{} Use COM instead of COG

\item {} 
\sphinxstyleliteralstrong{\sphinxupquote{mpifind}} (\sphinxstyleliteralemphasis{\sphinxupquote{bool}}) \textendash{} Use find\_monomer\_triplets\_MPI

\item {} 
\sphinxstyleliteralstrong{\sphinxupquote{signature\_one}} \textendash{} Atomic numbers of monomer one, format \{Z1 Z2 Z3 …\}

\item {} 
\sphinxstyleliteralstrong{\sphinxupquote{signature\_two}} \textendash{} Atomic numbers of monomer two, format \{Z1 Z2 Z3 …\}

\item {} 
\sphinxstyleliteralstrong{\sphinxupquote{signature\_three}} \textendash{} Atomic numbers of monomer three, format \{Z1 Z2 Z3 …\}

\item {} 
\sphinxstyleliteralstrong{\sphinxupquote{power}} (\sphinxstyleliteralemphasis{\sphinxupquote{float}}) \textendash{} Power of distances to be used in the kernel

\end{itemize}

\end{description}\end{quote}

\end{fulllineitems}

\index{molecule\_lo\_d (class in gap.descriptors)@\spxentry{molecule\_lo\_d}\spxextra{class in gap.descriptors}}

\begin{fulllineitems}
\phantomsection\label{\detokenize{descriptors:gap.descriptors.molecule_lo_d}}\pysiglinewithargsret{\sphinxbfcode{\sphinxupquote{class }}\sphinxcode{\sphinxupquote{gap.descriptors.}}\sphinxbfcode{\sphinxupquote{molecule\_lo\_d}}}{\emph{cutoff=0.0}, \emph{atoms\_template\_string=''}, \emph{neighbour\_graph\_depth=2}, \emph{signature=''}, \emph{symmetry\_property\_name='symm'}, \emph{append\_file=''}, \emph{atom\_ordercheck=True}, \emph{desctype=0}}{}~\begin{quote}\begin{description}
\item[{Parameters}] \leavevmode\begin{itemize}
\item {} 
\sphinxstyleliteralstrong{\sphinxupquote{cutoff}} (\sphinxstyleliteralemphasis{\sphinxupquote{float}}) \textendash{} Cutoff for molecule\_lo\_d-type descriptors

\item {} 
\sphinxstyleliteralstrong{\sphinxupquote{atoms\_template\_string}} \textendash{} Atoms object which serves as a template - written to a string

\item {} 
\sphinxstyleliteralstrong{\sphinxupquote{neighbour\_graph\_depth}} (\sphinxstyleliteralemphasis{\sphinxupquote{int}}) \textendash{} Ignore distances between atoms separated by more than this number of bonds

\item {} 
\sphinxstyleliteralstrong{\sphinxupquote{signature}} (\sphinxstyleliteralemphasis{\sphinxupquote{str}}) \textendash{} Atomic numbers of monomer one, format \{Z1 Z2 Z3 …\}

\item {} 
\sphinxstyleliteralstrong{\sphinxupquote{symmetry\_property\_name}} (\sphinxstyleliteralemphasis{\sphinxupquote{str}}) \textendash{} Integer arrays specifying symmetries - see header of make\_permutations\_v2.f95 for format

\item {} 
\sphinxstyleliteralstrong{\sphinxupquote{append\_file}} (\sphinxstyleliteralemphasis{\sphinxupquote{str}}) \textendash{} Pairs of atoms for which we want the distance to be additionally included in the descriptor

\item {} 
\sphinxstyleliteralstrong{\sphinxupquote{atom\_ordercheck}} (\sphinxstyleliteralemphasis{\sphinxupquote{bool}}) \textendash{} T: basic check that atoms in same order as in template F: assume all xyz frames have atoms in same order

\item {} 
\sphinxstyleliteralstrong{\sphinxupquote{desctype}} (\sphinxstyleliteralemphasis{\sphinxupquote{int}}) \textendash{} 0: distance matrix, 1: inverse distance matrix, 2: Coulomb matrix, 3: exponential

\end{itemize}

\end{description}\end{quote}

\end{fulllineitems}

\index{power\_so3 (class in gap.descriptors)@\spxentry{power\_so3}\spxextra{class in gap.descriptors}}

\begin{fulllineitems}
\phantomsection\label{\detokenize{descriptors:gap.descriptors.power_so3}}\pysiglinewithargsret{\sphinxbfcode{\sphinxupquote{class }}\sphinxcode{\sphinxupquote{gap.descriptors.}}\sphinxbfcode{\sphinxupquote{power\_so3}}}{\emph{cutoff=0.0}, \emph{min\_cutoff=0.0}, \emph{l\_max=4}, \emph{n\_max=4}, \emph{Z=0}, \emph{n\_species=1}, \emph{species\_Z=None}, \emph{w=None}}{}~\begin{quote}\begin{description}
\item[{Parameters}] \leavevmode\begin{itemize}
\item {} 
\sphinxstyleliteralstrong{\sphinxupquote{cutoff}} (\sphinxstyleliteralemphasis{\sphinxupquote{float}}) \textendash{} Cutoff for power\_so3-type descriptors

\item {} 
\sphinxstyleliteralstrong{\sphinxupquote{min\_cutoff}} (\sphinxstyleliteralemphasis{\sphinxupquote{float}}) \textendash{} Cutoff for minimal distances in power\_so3-type descriptors

\item {} 
\sphinxstyleliteralstrong{\sphinxupquote{l\_max}} (\sphinxstyleliteralemphasis{\sphinxupquote{int}}) \textendash{} L\_max for power\_so3-type descriptors

\item {} 
\sphinxstyleliteralstrong{\sphinxupquote{n\_max}} (\sphinxstyleliteralemphasis{\sphinxupquote{int}}) \textendash{} N\_max for power\_so3-type descriptors

\item {} 
\sphinxstyleliteralstrong{\sphinxupquote{Z}} (\sphinxstyleliteralemphasis{\sphinxupquote{int}}) \textendash{} Atomic number of central atom

\item {} 
\sphinxstyleliteralstrong{\sphinxupquote{n\_species}} (\sphinxstyleliteralemphasis{\sphinxupquote{int}}) \textendash{} Number of species for the descriptor

\item {} 
\sphinxstyleliteralstrong{\sphinxupquote{species\_Z}} (\sphinxstyleliteralemphasis{\sphinxupquote{int \_\_OR\_\_ ?}}\sphinxstyleliteralemphasis{\sphinxupquote{)}}) \textendash{} Atomic number of species

\item {} 
\sphinxstyleliteralstrong{\sphinxupquote{w}} (\sphinxstyleliteralemphasis{\sphinxupquote{float}}\sphinxstyleliteralemphasis{\sphinxupquote{) }}\sphinxstyleliteralemphasis{\sphinxupquote{\_\_OR\_\_ ?}}) \textendash{} Weight associated to each atomic type

\end{itemize}

\end{description}\end{quote}

\end{fulllineitems}

\index{power\_so4 (class in gap.descriptors)@\spxentry{power\_so4}\spxextra{class in gap.descriptors}}

\begin{fulllineitems}
\phantomsection\label{\detokenize{descriptors:gap.descriptors.power_so4}}\pysigline{\sphinxbfcode{\sphinxupquote{class }}\sphinxcode{\sphinxupquote{gap.descriptors.}}\sphinxbfcode{\sphinxupquote{power\_so4}}}
Args:
\index{power\_so4.soap (class in gap.descriptors)@\spxentry{power\_so4.soap}\spxextra{class in gap.descriptors}}

\begin{fulllineitems}
\phantomsection\label{\detokenize{descriptors:gap.descriptors.power_so4.soap}}\pysiglinewithargsret{\sphinxbfcode{\sphinxupquote{class }}\sphinxbfcode{\sphinxupquote{soap}}}{\emph{cutoff}, \emph{n\_max}, \emph{l\_max}, \emph{atom\_sigma}, \emph{cutoff\_transition\_width=0.5}, \emph{cutoff\_dexp=0}, \emph{cutoff\_scale=1.0}, \emph{cutoff\_rate=1.0}, \emph{central\_weight=1.0}, \emph{central\_reference\_all\_species=False}, \emph{average=False}, \emph{diagonal\_radial=False}, \emph{covariance\_sigma0=0.0}, \emph{normalise=True}, \emph{basis\_error\_exponent=10.0}, \emph{n\_Z=1}, \emph{n\_species=1}, \emph{xml\_version=1426512068}, \emph{species\_Z=None}, \emph{Z=0}}{}~\begin{quote}\begin{description}
\item[{Parameters}] \leavevmode\begin{itemize}
\item {} 
\sphinxstyleliteralstrong{\sphinxupquote{species\_Z}} \textendash{} Atomic number of species

\item {} 
\sphinxstyleliteralstrong{\sphinxupquote{Z}} (\sphinxstyleliteralemphasis{\sphinxupquote{int}}) \textendash{} Atomic number of central atom, 0 is the wild-card \_\_OR\_\_ Atomic numbers to be considered for central atom, must be a list

\item {} 
\sphinxstyleliteralstrong{\sphinxupquote{cutoff}} \textendash{} Cutoff for soap-type descriptors

\item {} 
\sphinxstyleliteralstrong{\sphinxupquote{atom\_sigma}} \textendash{} Width of atomic Gaussians for soap-type descriptors

\item {} 
\sphinxstyleliteralstrong{\sphinxupquote{n\_max}} \textendash{} N\_max (number of radial basis functions) for soap-type descriptors

\item {} 
\sphinxstyleliteralstrong{\sphinxupquote{l\_max}} \textendash{} L\_max (spherical harmonics basis band limit) for soap-type descriptors

\item {} 
\sphinxstyleliteralstrong{\sphinxupquote{n\_Z}} (\sphinxstyleliteralemphasis{\sphinxupquote{int}}) \textendash{} How many different types of central atoms to consider

\item {} 
\sphinxstyleliteralstrong{\sphinxupquote{n\_species}} (\sphinxstyleliteralemphasis{\sphinxupquote{int}}) \textendash{} Number of species for the descriptor

\item {} 
\sphinxstyleliteralstrong{\sphinxupquote{cutoff\_transition\_width}} (\sphinxstyleliteralemphasis{\sphinxupquote{float}}) \textendash{} Cutoff transition width for soap-type descriptors

\item {} 
\sphinxstyleliteralstrong{\sphinxupquote{cutoff\_dexp}} (\sphinxstyleliteralemphasis{\sphinxupquote{int}}) \textendash{} Cutoff decay exponent

\item {} 
\sphinxstyleliteralstrong{\sphinxupquote{cutoff\_scale}} (\sphinxstyleliteralemphasis{\sphinxupquote{float}}) \textendash{} Cutoff decay scale

\item {} 
\sphinxstyleliteralstrong{\sphinxupquote{cutoff\_rate}} (\sphinxstyleliteralemphasis{\sphinxupquote{float}}) \textendash{} Inverse cutoff decay rate

\item {} 
\sphinxstyleliteralstrong{\sphinxupquote{central\_weight}} (\sphinxstyleliteralemphasis{\sphinxupquote{float}}) \textendash{} Weight of central atom in environment

\item {} 
\sphinxstyleliteralstrong{\sphinxupquote{central\_reference\_all\_species}} (\sphinxstyleliteralemphasis{\sphinxupquote{bool}}) \textendash{} Place a Gaussian reference for all atom species densities. By default (F) only consider when neighbour is the same species as centre

\item {} 
\sphinxstyleliteralstrong{\sphinxupquote{covariance\_sigma0}} (\sphinxstyleliteralemphasis{\sphinxupquote{float}}) \textendash{} sigma\_0 parameter in polynomial covariance function

\item {} 
\sphinxstyleliteralstrong{\sphinxupquote{xml\_version}} (\sphinxstyleliteralemphasis{\sphinxupquote{int}}) \textendash{} Version of GAP the XML potential file was created

\item {} 
\sphinxstyleliteralstrong{\sphinxupquote{basis\_error\_exponent}} (\sphinxstyleliteralemphasis{\sphinxupquote{float}}) \textendash{} 10\textasciicircum{}(-basis\_error\_exponent) is the max difference between the target and the expanded function

\item {} 
\sphinxstyleliteralstrong{\sphinxupquote{average}} (\sphinxstyleliteralemphasis{\sphinxupquote{bool}}) \textendash{} Whether to calculate averaged SOAP - one descriptor per atoms object. If false (default) atomic SOAP is returned.

\item {} 
\sphinxstyleliteralstrong{\sphinxupquote{diagonal\_radial}} (\sphinxstyleliteralemphasis{\sphinxupquote{bool}}) \textendash{} Only return the n1=n2 elements of the power spectrum.

\item {} 
\sphinxstyleliteralstrong{\sphinxupquote{normalise}} (\sphinxstyleliteralemphasis{\sphinxupquote{bool}}) \textendash{} Normalise descriptor so magnitude is 1. In this case the kernel of two equivalent environments is 1.

\end{itemize}

\end{description}\end{quote}

\end{fulllineitems}


\end{fulllineitems}

\index{rdf (class in gap.descriptors)@\spxentry{rdf}\spxextra{class in gap.descriptors}}

\begin{fulllineitems}
\phantomsection\label{\detokenize{descriptors:gap.descriptors.rdf}}\pysiglinewithargsret{\sphinxbfcode{\sphinxupquote{class }}\sphinxcode{\sphinxupquote{gap.descriptors.}}\sphinxbfcode{\sphinxupquote{rdf}}}{\emph{cutoff=0.0}, \emph{transition\_width=0.2}, \emph{Z=0}, \emph{r\_min=0.0}, \emph{r\_max=0.0}, \emph{n\_gauss=10}, \emph{w\_gauss=0.0}}{}~\begin{quote}\begin{description}
\item[{Parameters}] \leavevmode\begin{itemize}
\item {} 
\sphinxstyleliteralstrong{\sphinxupquote{cutoff}} (\sphinxstyleliteralemphasis{\sphinxupquote{float}}) \textendash{} Cutoff for rdf-type descriptors

\item {} 
\sphinxstyleliteralstrong{\sphinxupquote{transition\_width}} (\sphinxstyleliteralemphasis{\sphinxupquote{float}}) \textendash{} Width of transition region from 1 to 0

\item {} 
\sphinxstyleliteralstrong{\sphinxupquote{Z}} (\sphinxstyleliteralemphasis{\sphinxupquote{int}}) \textendash{} Atomic number of central atom

\item {} 
\sphinxstyleliteralstrong{\sphinxupquote{r\_min}} (\sphinxstyleliteralemphasis{\sphinxupquote{float}}) \textendash{} Atomic number of central atom

\item {} 
\sphinxstyleliteralstrong{\sphinxupquote{r\_max}} (\sphinxstyleliteralemphasis{\sphinxupquote{float}}) \textendash{} Atomic number of central atom

\item {} 
\sphinxstyleliteralstrong{\sphinxupquote{n\_gauss}} (\sphinxstyleliteralemphasis{\sphinxupquote{int}}) \textendash{} Atomic number of central atom

\item {} 
\sphinxstyleliteralstrong{\sphinxupquote{w\_gauss}} (\sphinxstyleliteralemphasis{\sphinxupquote{float}}) \textendash{} Atomic number of central atom

\end{itemize}

\end{description}\end{quote}

\end{fulllineitems}

\index{sparse\_method (class in gap.descriptors)@\spxentry{sparse\_method}\spxextra{class in gap.descriptors}}

\begin{fulllineitems}
\phantomsection\label{\detokenize{descriptors:gap.descriptors.sparse_method}}\pysigline{\sphinxbfcode{\sphinxupquote{class }}\sphinxcode{\sphinxupquote{gap.descriptors.}}\sphinxbfcode{\sphinxupquote{sparse\_method}}}
sparse\_method options are:
\begin{itemize}
\item {} 
\sphinxstylestrong{RANDOM}: default, chooses n\_sparse random datapoints

\item {} 
\sphinxstylestrong{PIVOT}: based on the full covariance matrix finds the n\_sparse “pivoting” points

\item {} 
\sphinxstylestrong{CLUSTER}: based on the full covariance matrix performs a k-medoid clustering into n\_sparse clusters, returning the medoids

\item {} 
\sphinxstylestrong{UNIFORM}: makes a histogram of the data based on n\_sparse and returns a data point from each bin

\item {} 
\sphinxstylestrong{KMEANS}: k-means clustering based on the data points

\item {} 
\sphinxstylestrong{COVARIANCE}: greedy data point selection based on the sparse covariance matrix, to minimise the GP variance of all datapoints

\item {} 
\sphinxstylestrong{UNIQ}: selects unique datapoints from the dataset

\item {} 
\sphinxstylestrong{FUZZY}: fuzzy k-means clustering

\item {} 
\sphinxstylestrong{FILE}: reads sparse points from a file

\item {} 
\sphinxstylestrong{INDEX\_FILE}: reads indices of sparse points from a file

\item {} 
\sphinxstylestrong{CUR\_COVARIANCE}: CUR, based on the full covariance matrix

\item {} 
\sphinxstylestrong{CUR\_POINTS}: CUR, based on the datapoints

\end{itemize}

\end{fulllineitems}

\index{trihis (class in gap.descriptors)@\spxentry{trihis}\spxextra{class in gap.descriptors}}

\begin{fulllineitems}
\phantomsection\label{\detokenize{descriptors:gap.descriptors.trihis}}\pysiglinewithargsret{\sphinxbfcode{\sphinxupquote{class }}\sphinxcode{\sphinxupquote{gap.descriptors.}}\sphinxbfcode{\sphinxupquote{trihis}}}{\emph{cutoff=0.0}, \emph{n\_gauss=0}, \emph{trihis\_gauss\_centre=None}, \emph{trihis\_gauss\_width=None}}{}~\begin{quote}\begin{description}
\item[{Parameters}] \leavevmode\begin{itemize}
\item {} 
\sphinxstyleliteralstrong{\sphinxupquote{cutoff}} (\sphinxstyleliteralemphasis{\sphinxupquote{float}}) \textendash{} Cutoff for trihis-type descriptors

\item {} 
\sphinxstyleliteralstrong{\sphinxupquote{n\_gauss}} (\sphinxstyleliteralemphasis{\sphinxupquote{int}}) \textendash{} Number of Gaussians for trihis-type descriptors

\item {} 
\sphinxstyleliteralstrong{\sphinxupquote{trihis\_gauss\_centre}} \textendash{} Number of Gaussians for trihis-type descriptors

\item {} 
\sphinxstyleliteralstrong{\sphinxupquote{trihis\_gauss\_width}} \textendash{} Number of Gaussians for trihis-type descriptors

\end{itemize}

\end{description}\end{quote}

\end{fulllineitems}

\index{water\_dimer (class in gap.descriptors)@\spxentry{water\_dimer}\spxextra{class in gap.descriptors}}

\begin{fulllineitems}
\phantomsection\label{\detokenize{descriptors:gap.descriptors.water_dimer}}\pysiglinewithargsret{\sphinxbfcode{\sphinxupquote{class }}\sphinxcode{\sphinxupquote{gap.descriptors.}}\sphinxbfcode{\sphinxupquote{water\_dimer}}}{\emph{cutoff=0.0}, \emph{cutoff\_transition\_width=0.5}, \emph{monomer\_cutoff=1.5}, \emph{OHH\_ordercheck=True}, \emph{power=1.0}}{}~\begin{quote}\begin{description}
\item[{Parameters}] \leavevmode\begin{itemize}
\item {} 
\sphinxstyleliteralstrong{\sphinxupquote{cutoff}} (\sphinxstyleliteralemphasis{\sphinxupquote{float}}) \textendash{} Cutoff for water\_dimer-type descriptors

\item {} 
\sphinxstyleliteralstrong{\sphinxupquote{cutoff\_transition\_width}} (\sphinxstyleliteralemphasis{\sphinxupquote{float}}) \textendash{} Width of smooth cutoff region for water\_dimer-type descriptors

\item {} 
\sphinxstyleliteralstrong{\sphinxupquote{monomer\_cutoff}} (\sphinxstyleliteralemphasis{\sphinxupquote{float}}) \textendash{} Monomer cutoff for water\_dimer-type descriptors

\item {} 
\sphinxstyleliteralstrong{\sphinxupquote{OHH\_ordercheck}} (\sphinxstyleliteralemphasis{\sphinxupquote{bool}}) \textendash{} T: find water molecules. F: use default order OHH

\item {} 
\sphinxstyleliteralstrong{\sphinxupquote{power}} (\sphinxstyleliteralemphasis{\sphinxupquote{float}}) \textendash{} Power of distances to be used in the kernel

\end{itemize}

\end{description}\end{quote}

\end{fulllineitems}

\index{water\_monomer (class in gap.descriptors)@\spxentry{water\_monomer}\spxextra{class in gap.descriptors}}

\begin{fulllineitems}
\phantomsection\label{\detokenize{descriptors:gap.descriptors.water_monomer}}\pysiglinewithargsret{\sphinxbfcode{\sphinxupquote{class }}\sphinxcode{\sphinxupquote{gap.descriptors.}}\sphinxbfcode{\sphinxupquote{water\_monomer}}}{\emph{cutoff=0.0}}{}~\begin{quote}\begin{description}
\item[{Parameters}] \leavevmode
\sphinxstyleliteralstrong{\sphinxupquote{cutoff}} (\sphinxstyleliteralemphasis{\sphinxupquote{float}}) \textendash{} Cutoff for water\_monomer-type descriptors

\end{description}\end{quote}

\end{fulllineitems}



\section{Using Potentials}
\label{\detokenize{potentials:using-potentials}}\label{\detokenize{potentials::doc}}

\subsection{QUIP}
\label{\detokenize{potentials:quip}}
QUantum mechanics and Interatomic Potentials (QUIP)


\subsubsection{Potentials implemented in \sphinxstyleliteralintitle{\sphinxupquote{QUIP}}}
\label{\detokenize{potentials:potentials-implemented-in-quip}}
Classical interatomic potentials:
\begin{itemize}
\item {} 
BKS (silica)

\item {} 
Brenner (carbon)

\item {} 
EAM (fcc)

\item {} 
Fanourgakis-Xantheas

\item {} 
Finnis-Sinclair (bcc)

\item {} 
Flikkema-Bromley

\item {} 
GAP (general many-body)

\item {} 
Guggenheim-McGlashan

\item {} 
Lennard-Jones

\item {} 
Morse

\item {} 
Partridge-Schwenke (water monomer)

\item {} 
Si-MEAM (silicon)

\item {} 
Stillinger-Weber (carbon, silicon, germanium)

\item {} 
Stillinger-Weber + Vashishta (silicon/silica interfaces)

\item {} 
Sutton-Chen

\item {} 
Tangney-Scandolo (silica, titania etc)

\item {} 
Tersoff (silicon, carbon)

\end{itemize}

Plus several tight binding parameterisations (Bowler, DFTB, GSP,
NRL-TB, …)

External packages:
\begin{itemize}
\item {} 
\sphinxcode{\sphinxupquote{CASTEP}}— DFT, planewaves, ultrasoft pseudopotentials

\item {} 
\sphinxcode{\sphinxupquote{CP2K}} — DFT, mixed Gaussian/planewave basis set, various pseudopotentials.
Our \sphinxcode{\sphinxupquote{CP2K}} Driver supports QM, MM and QM/MM.

\item {} 
\sphinxcode{\sphinxupquote{MOLPRO}} — All electron quantum chemistry code. DFT, CCSD(T), MP2

\item {} 
\sphinxcode{\sphinxupquote{VASP}} — DFT, planewaves, PAW or ultrasoft pseudopotentials

\item {} 
Interface to \sphinxhref{http://www.openkim.org}{OpenKIM} project

\item {} 
Relatively easy to add new codes

\end{itemize}

\sphinxcode{\sphinxupquote{QUIP}} also has a full interface to the Atomic Simulation
Environment, \sphinxhref{https://wiki.fysik.dtu.dk/ase}{ASE}
\begin{itemize}
\item {} 
\sphinxcode{\sphinxupquote{ASE}} adds support for several more codes e.g. \sphinxcode{\sphinxupquote{ABINIT}}, \sphinxcode{\sphinxupquote{Elk}},
\sphinxcode{\sphinxupquote{Exciting}}, \sphinxcode{\sphinxupquote{GPAW}}, \sphinxcode{\sphinxupquote{SIESTA}}, …

\item {} 
\sphinxcode{\sphinxupquote{ASE}} also works with the \sphinxhref{https://wiki.fysik.dtu.dk/cmr}{CMR} database system

\end{itemize}


\subsubsection{Performing calculations}
\label{\detokenize{potentials:performing-calculations}}\begin{itemize}
\item {} 
As well as preparing structures and post-processing results,
\sphinxcode{\sphinxupquote{quippy}} allows calculations to be run

\item {} 
In \sphinxcode{\sphinxupquote{QUIP}} and \sphinxcode{\sphinxupquote{quippy}}, all calculations are performed with a
Potential object (very similar to the
\sphinxcode{\sphinxupquote{Calculator}} concept in \sphinxcode{\sphinxupquote{ASE}})

\item {} 
Types of potential
\begin{itemize}
\item {} 
\sphinxstyleemphasis{Internal}: interatomic potential or tight binding

\item {} 
\sphinxstyleemphasis{External}: file-based communication with external code or callback-based communication with a Python function

\item {} 
Plus flexible combinations of other potentials

\end{itemize}

\item {} 
\sphinxstyleemphasis{Internal} potentials use XML parameter strings

\item {} 
\sphinxstyleemphasis{External} potentials use template parameter files

\end{itemize}


\subsubsection{Creating a Potential}
\label{\detokenize{potentials:creating-a-potential}}
Internal potential:

\begin{sphinxVerbatim}[commandchars=\\\{\}]
\PYG{g+gp}{\PYGZgt{}\PYGZgt{}\PYGZgt{} }\PYG{n}{sw\PYGZus{}pot} \PYG{o}{=} \PYG{n}{Potential}\PYG{p}{(}\PYG{l+s+s1}{\PYGZsq{}}\PYG{l+s+s1}{IP SW}\PYG{l+s+s1}{\PYGZsq{}}\PYG{p}{)}
\end{sphinxVerbatim}

External potential:

\begin{sphinxVerbatim}[commandchars=\\\{\}]
\PYG{g+gp}{\PYGZgt{}\PYGZgt{}\PYGZgt{} }\PYG{n}{castep} \PYG{o}{=} \PYG{n}{Potential}\PYG{p}{(}\PYG{l+s+s1}{\PYGZsq{}}\PYG{l+s+s1}{FilePot}\PYG{l+s+s1}{\PYGZsq{}}\PYG{p}{,}
\PYG{g+gp}{... }                   \PYG{n}{command}\PYG{o}{=}\PYG{l+s+s1}{\PYGZsq{}}\PYG{l+s+s1}{./castep\PYGZhy{}driver.sh}\PYG{l+s+s1}{\PYGZsq{}}\PYG{p}{)}
\end{sphinxVerbatim}

Driver script can be a shell script, an executable program using
\sphinxcode{\sphinxupquote{QUIP}} or a \sphinxcode{\sphinxupquote{quippy}} script. It can even invoke code on a remote
machine.


\subsection{GAP}
\label{\detokenize{potentials:gap}}\begin{description}
\item[{Example:}] \leavevmode
sw\_pot = Potential(‘IP SW’)

\end{description}


\subsubsection{XML functions}
\label{\detokenize{potentials:xml-functions}}

\section{Tutorials}
\label{\detokenize{tutorials:tutorials}}\label{\detokenize{tutorials::doc}}
How to use GAP.


\subsection{Descriptors}
\label{\detokenize{tutorials:descriptors}}

\subsection{Learning charges}
\label{\detokenize{tutorials:learning-charges}}

\subsection{Si surface}
\label{\detokenize{tutorials:si-surface}}

\subsection{Crack propagation}
\label{\detokenize{tutorials:crack-propagation}}

\chapter{Indices and tables}
\label{\detokenize{index:indices-and-tables}}\begin{itemize}
\item {} 
\DUrole{xref,std,std-ref}{genindex}

\item {} 
\DUrole{xref,std,std-ref}{modindex}

\item {} 
\DUrole{xref,std,std-ref}{search}

\end{itemize}


\section{References}
\label{\detokenize{index:references}}
\begin{sphinxthebibliography}{LOTF}
\bibitem[LOTF]{index:lotf}
Csányi, G., Albaret, T., Payne, M., \& De Vita,
A. ‘Learn on the Fly’: A Hybrid Classical and Quantum-Mechanical
Molecular Dynamics Simulation. Physical Review Letters,
93(17), 175503. (2004) \sphinxurl{http://prl.aps.org/abstract/PRL/v93/i17/e175503}\textgreater{}
\end{sphinxthebibliography}


\renewcommand{\indexname}{Python Module Index}
\begin{sphinxtheindex}
\let\bigletter\sphinxstyleindexlettergroup
\bigletter{g}
\item\relax\sphinxstyleindexentry{gap}\sphinxstyleindexpageref{gap:\detokenize{module-gap}}
\item\relax\sphinxstyleindexentry{gap.descriptors}\sphinxstyleindexpageref{descriptors:\detokenize{module-gap.descriptors}}
\indexspace
\bigletter{q}
\item\relax\sphinxstyleindexentry{quippy}\sphinxstyleindexpageref{index:\detokenize{module-quippy}}
\end{sphinxtheindex}

\renewcommand{\indexname}{Index}
\printindex
\end{document}